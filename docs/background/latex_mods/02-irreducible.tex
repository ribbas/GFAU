\subsection{Determining Irreducibility} A polynomial is said to be
irreducible if and only if there exists no roots for it.

    \subsubsection{Analytical Approach} If the sum of the
    coefficients of the polynomial equals 1 when $x=0$ and $x=1$, the
    polynomial is irreducible.

    \paragraph{Example} \leavevmode \\ Show that the polynomial
    \examplepoly~ is irreducible in $GF(2)[x]$ using the mathematical
    algorithm:
        \begin{equation*}
            \begin{split}
                x=0 : \ &  0^{3} + 0^{2} + 0^{0} \\
                & = 0 + 0 + 1 \\
                & = 1 \\
                x=1 : \ &  1^{3} + 1^{2} + 1^{0} \\
                & = 1 + 1 + 1 \\
                & = 1 \\
            \end{split}
        \end{equation*}

        \centerline{$\therefore$ \examplepoly~ has no roots in
        $GF(2)[x]$.}

        Since the sum of the coefficients is 1 for both the cases, the
        polynomial is therefore determined irreducible.

    \subsubsection{Digital Logic} To design such algorithm, the
    polynomial may be checked for the following attributes:
    \begin{enumerate}
        \item the coefficient of its 0th term is 1
        \item the total number of non-zero coefficients is odd
    \end{enumerate}

    If both of these conditions are met, the polynomial is irreducible.

    \paragraph{Example} \leavevmode \\ Show that the polynomial
    \examplepoly~ is irreducible using the digital algorithm:

        \[ <0000 \ 0000 \ 0000 \ 1101> \]

    The 0th bit is 1 and the total number of non-zero coefficients is
    3. Therefore, the polynomial is determined irreducible.
