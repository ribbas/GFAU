\section{Background} A Galois Field is a field with a finite number of
elements. The nomenclature $GF(q)$ is used to indicate a Galois Field with
$q$ elements. In $GF(q)$ the parameter $q$ must be a power of a prime. For
each prime power there exists exactly one finite field. The binary field
$GF(2)$ is the most frequently used Galois field.

    \subsection{Purpose and Scope} The Galois Field Arithmetic Unit
    operates in fields of $q^n$, where $2 \leq n \leq 16$. This document
    will demonstrate the functionality of the unit by deriving all its
    functionality through mathematical approaches. The mathematical
    algorithms will be followed by their corresponding implementations in
    digital design.

    \subsection{Terms and Keywords}

        \subsubsection{Input Primitive Polynomials} Input primiteive
        polynomials in the Galois Field are represented as:

        \[ c_{n}x^{n}+\ldots+c_{2}x^{2}+c_{1}x^{1}+c_{0}x^{0}, \ where \
        c,x \ \epsilon \ GF(2)=\{0,1\} \]

        For convenience and simplicity, all the examples provided will
        refer to the following polynomial: \examplepoly~.

        \subsubsection{Elements} The elements of an input polynomial refer
        to the $2^{n}-1$ elements in the field.

        \subsubsection{Polynomial Form} The polynomial forms of the
        elements refer to the $2^{n}-1$ symbolic representations of the of
        an input primitive polynomials in the field.

    \newpage
