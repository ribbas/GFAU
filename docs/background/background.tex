%%%%%%%%%%%%%%%%%%%%%%%%%%%%%%%%%%%%%%%%%
% Template LaTeX Template Version 1.0 (December 8 2014)
%
% This template has been downloaded from: http://www.LaTeXTemplates.com
%
% Original author: Brandon Fryslie With extensive modifications by: Vel
% (vel@latextemplates.com)
%
% License: CC BY-NC-SA 3.0 (http://creativecommons.org/licenses/by-nc-sa/3.0/)
%
% Author: Sabbir Ahmed
% 
%%%%%%%%%%%%%%%%%%%%%%%%%%%%%%%%%%%%%%%%%

\documentclass[11pt]{extarticle}
%%%%%%%%%%%%%%%%%%%%%%%%%%%%%%%%%%%%%%%%%
% Structure Structural Definitions File Version 1.0 (December 8 2014)
%
% Created by: Vel (vel@latextemplates.com)
% 
% This file has been downloaded from: http://www.LaTeXTemplates.com
%
% License: CC BY-NC-SA 3.0 (http://creativecommons.org/licenses/by-nc-sa/3.0/)
%
%%%%%%%%%%%%%%%%%%%%%%%%%%%%%%%%%%%%%%%%%

\usepackage{geometry}  % page layout modification

\usepackage[utf8]{inputenc}  % letters with accents
\usepackage[T1]{fontenc}  % use 8-bit encoding that has 256 glyphs
\usepackage{avant}  % Avantgarde font

\usepackage{amsmath}  % equations
\usepackage{amssymb}  % extention of symbols
\usepackage{bm}  % extention of bolding features
\usepackage{caption}  % captions for tables
\usepackage{enumitem}  % extra options for enumeration
\usepackage[pdftex]{graphicx}  % figures
\usepackage{setspace}  % extension of line spacing options

\usepackage{enumitem}

\usepackage[square,numbers]{natbib}  % for bibtex

% add more sections beyond \subsubsection
\setcounter{tocdepth}{4}
\setcounter{secnumdepth}{4}

\setlength{\parindent}{0mm} % don't indent paragraphs
\setlength{\parskip}{2.5mm} % space between paragraphs
\renewcommand{\baselinestretch}{1.5}  % space between lines

\setlength{\textwidth}{16cm} % width of the text on the page
\setlength{\textheight}{23cm} % height of the text on the page
\setlength{\oddsidemargin}{0cm} % width of the margin
\setlength{\topmargin}{-1.25cm} % reduce the top margin

\captionsetup[table]{skip=10pt}  % space between tables and their captions
\captionsetup{labelfont=bf}  % bold captions

\renewcommand\familydefault{\sfdefault}  % default font for entire document
 % specifies the document layout and style

% names
\newcommand{\team}{Galois Field Arithmetic Unit}
\newcommand{\Sabbir}{Sabbir Ahmed}
\newcommand{\Jeffrey}{Jeffrey Osazuwa}
\newcommand{\Howard}{Howard To}
\newcommand{\Brian}{Brian Weber}
\newcommand{\examplepoly}{$x^{3}+x^{2}+x^{0}$}

% document info command
\newcommand{\documentinfo}[4]{
    \begin{centering}
        \parbox{2in}{
        \begin{spacing}{1}
            \begin{flushleft}
                \begin{tabular}{l l} #1 \\ #2 \\ #3 \\ #4 \\
                \end{tabular} \\
                \rule{\textwidth}{1pt}
            \end{flushleft}
        \end{spacing} }
    \end{centering} }

\begin{document}

    \documentinfo{\textbf{MEMO NAME:} GFAU\_BACKGROUND}{\textbf{DATE:}
    \today}{\textbf{TO:} EFC LaBerge}{\textbf{SUBJECT: } Background on the
    Galois Field Arithmetic Unit}
    \vspace{-0.1in}

    % \tableofcontents
    \section{Background} Galois fields (denoted $\gf{p^n}$, where $p,n \in
\mathbb{N}_{>0}$, and $p$ is prime) are fields with finite orders of $p^n$.
They are a key part of number theory, abstract algebra, arithmetic algebraic
geometry, and cryptography. In error detection and correction, Galois fields
are utilized in cyclic redundancy check (CRC) which are used in digital
networks and storage devices to detect accidental changes to raw data.

The Galois Field Arithmetic Unit (GFAU) is a scalable arithmetic logic unit
(ALU) capable of generating elements in the Galois field of an irreducible
polynomial in $\gf{2^n}$, where $2 \leq n < (\text{$\#$ of memory address
pins})$. GFAU supports addition, subtraction, multiplication, division and
logarithm of these elements for low powered devices.

    \subsection{Purpose and Scope} This document will demonstrate the
    functionality of the unit by deriving all its functionality through
    mathematical approaches. The mathematical algorithms will be accompanied by
    the modules that implement them in digital design. The modules will contain
    brief documentation on their usages.

    \subsection{Terms and Keywords}

        \subsubsection{Input Primitive Polynomials} Input primitive
        polynomials in the Galois Field are represented as:

        \[ c_{n}x^{n}+\ldots+c_{2}x^{2}+c_{1}x^{1}+c_{0}x^{0}, \text{ where }
        c,x \in \mathbb{Z}_2 \]

        For convenience and simplicity, all the examples provided will refer to
        the polynomial: \examplepoly~.

        \subsubsection{Elements} The elements of an input polynomial refer
        to the $2^{n}-1$ elements in the field.

        \subsubsection{Polynomial Form} The polynomial forms of the
        elements refer to the $2^{n}-1$ symbolic representations of the input
        primitive polynomials in the field.

        \subsubsection{Example} An example of the elements and their
        corresponding polynomials is provided below:

            \begin{table}[h]
                \def\arraystretch{1.5}
                \caption{The 8 Element Vectors of \examplepoly~ in $\gf{2}[x]$}

                \centering
                \begin{tabular*}{250pt}{@{\extracolsep{\fill}} c|c|c|c}

                \textbf{Element} & \textbf{Symbol} & \textbf{Polynomial Form} &
                \textbf{Symbol} \\
                \hline
                $0$ (NULL)  & {\scriptsize [1]} & $0+0+0$               & 000\\
                $\beta^{0}$ & 000 & $0 + 0 + \beta^{0}$                 & 001\\
                $\beta^{1}$ & 001 & $0 + \beta^{1} + 0$                 & 010\\
                $\beta^{2}$ & 010 & $\beta^{2} + 0 + 0$                 & 100\\
                $\beta^{3}$ & 011 & $\beta^{2} + 0 + \beta^{0}$         & 101\\
                $\beta^{4}$ & 100 & $\beta^{2} + \beta^{1} + \beta^{0}$ & 111\\
                $\beta^{5}$ & 101 & $0 + \beta^{1} + \beta^{0}$         & 011\\
                $\beta^{6}$ & 110 & $\beta^{2} + \beta^{1} + 0$         & 110\\
                $\beta^{7}$ & {\scriptsize [2]} & $0 + 0 + \beta^{0}$   & 001\\
                \end{tabular*}
            \end{table}

            {\scriptsize [1]} The additive identity, 0 (zero), referred to as
            NULL, in its element form is reserved where its binary symbol does
            not represent its decimal value. \\ {\scriptsize [2]} Elements
            beyond the $(2^{n}-1)$th element will be handled with special
            conditions since they cycle back to previous elements.

        \subsubsection{Notations} Several notations are used in this document
        to assist in linking the mathematical and digital design concepts. A
        list of selected notations are provided below:

        \begin{table}[!h]
            \def\arraystretch{2}
            \caption{Mathematical and Logical Notations}

            \centering
            \begin{tabular*}{470pt}{@{\extracolsep{\fill}} r p{9cm}}

            \textbf{Notation} & \textbf{Definition} \\
            \hline
            $a \ + \ b$ & Arithmetic addition of $a$ and $b$\\
            $a \ - \ b$ & Arithmetic subtraction of $a$ and $b$\\
            $a \ \cdot \ b$ & Arithmetic multiplication of $a$ and $b$\\
            $a \ / \ b$ & Arithmetic Division of $a$ and $b$\\
            $a \ \wedge \ b$ & Logical conjunction of $a$ and $b$\\
            $a \ \vee \ b$ & Logical disjunction of $a$ and $b$\\
            $a \ \oplus \ b$ & Logical exclusive disjunction of $a$ and $b$\\
            $\overline{a}$ & Logical complement of $a$\\
            $a \ \ll \ b$ & Logical shift-left $a$ by $b$ bits\\
            $a_{2's}$ & Two's complement of $a$\\
            $|a|$ & Number of bits in $a$\\
            $<x_{n}, \ x_{n-1}, \ \ldots, \ x_{0}>$ & Ordered array of size $n$
            \\
            $\{x_{n}, \ x_{n-1}, \ \ldots, \ x_{0}\}$ & Unordered set of size
            $n$ \\
            $< x_{m}, \ \ \overleftarrow{0_{m-1}, \ \ldots, \ 0_{n}}, x_{n-1},
            \
            \ldots, \ x_{0} >$ & Zero-padding between the $(m-1)$th and $n$th
            bits \\
            $A[i]$ & $i$th bit of array $A$ \\
            $A[i:j]$ & Subarray of array $A$ from its $i$th to $j$th index
            inclusive; where $i > j$ \\

            \end{tabular*}
        \end{table}

    \clearpage
    \newpage


    \section{Algorithms} Input polynomials will be represented as 16 bit zero-
    base arrays. For example, the polynomial \examplepoly~ will be represented
    as

        \[ <0000 \ 0000 \ 0000 \ 1101> (3rd, \ 2nd, \ and \ 0th \ bits) \]

        \subsection{Determining Irreducibility} A polynomial is said to be
irreducible if and only if there exists no roots for it.

    \subsubsection{Analytical Approach} If the sum of the
    coefficients of the polynomial equals 1 when $x=0$ and $x=1$, the
    polynomial is irreducible.

    \paragraph{Example} \leavevmode \\ Show that the polynomial
    \examplepoly~ is irreducible in $GF(2)[x]$ using the mathematical
    algorithm:
        \begin{equation*}
            \begin{split}
                x=0 : \ &  0^{3} + 0^{2} + 0^{0} \\
                & = 0 + 0 + 1 \\
                & = 1 \\
                x=1 : \ &  1^{3} + 1^{2} + 1^{0} \\
                & = 1 + 1 + 1 \\
                & = 1 \\
            \end{split}
        \end{equation*}

        \centerline{$\therefore$ \examplepoly~ has no roots in
        $GF(2)[x]$.}

        Since the sum of the coefficients is 1 for both the cases, the
        polynomial is therefore determined irreducible.

    \subsubsection{Digital Logic} To design such algorithm, the
    polynomial may be checked for the following attributes:
    \begin{enumerate}
        \item the coefficient of its 0th term is 1
        \item the total number of non-zero coefficients is odd
    \end{enumerate}

    If both of these conditions are met, the polynomial is irreducible.

    \paragraph{Example} \leavevmode \\ Show that the polynomial
    \examplepoly~ is irreducible using the digital algorithm:

        \[ <0000 \ 0000 \ 0000 \ 1101> \]

    The 0th bit is 1 and the total number of non-zero coefficients is
    3. Therefore, the polynomial is determined irreducible.

\newpage



        \subsection{Symbols}

    Once a polynomial is determined irreducible, its symbols may be generated.
    The number of terms grow exponentially, $2^{n}-1$, where $n$ is the highest
    degree of the polynomial.

    \subsubsection{Default Symbols}

    Default symbols refer to terms in the field that exist for Galois Fields of
    all irreducible polynomials of $q^n$, where $2 \leq n
    \leq 16$. Since the number of elements cannot be smaller than $2$, only
    zero and the 0th and 1st elements are shared among all fields.

    \begin{table}[h]
        \def\arraystretch{2.5}
        \caption{Default Symbols Generated for All Irreducible Polynomials}
        \centering
        \begin{tabular*}{300pt}{@{\extracolsep{\fill}} ccc}

        \textbf{Element} & \textbf{Polynomial Form} & \textbf{Symbol}
        \\
        \hline $0$ & $0_{15} + \ldots + 0_{2} + 0_{1} + 0_{0}$ & $\{ 0_{15}
        \ldots 0_{2} 0_{1} 0_{0} \}$ \\

        $\alpha^{0}$ & $0_{15} + \ldots + 0_{2} + 0_{1} +
        \alpha^{0}_{0}$ & $\{0_{15} \ldots 0_{2} 0_{1} 1_{0} \}$ \\

        $\alpha^{1}$ & $0_{15} + \ldots + 0_{2} + \alpha^{1}_{1} + 0$ & $\{
        0_{15} \ldots 0_{2} 1_{1} 0_{0} \}$ \\

        \end{tabular*}
        \label{table:default_sym}
    \end{table}

    Table \ref{table:default_sym} have all the bits of their values set to $0$
    except where indicated.

        \paragraph{Analytical Approach} \leavevmode\\ Using the example
        polynomial \examplepoly~, show that 0, the 0th element and the first
        element exist in $GF(2^{3})$.

        \hspace*{\fill}
        \centerline{Let $\beta \ \epsilon \ GF(2^{3})$ be a root of
        \examplepoly~ $\implies \beta^{3}+\beta^{2}+\beta^{0}$}

        \hspace*{\fill}
        \centerline{$\therefore$ The coefficients are in $GF(2) \implies
        \beta^{3}=\beta^{2}+\beta^{0}$}
        \[ \because a \ field \ has \ additive \ and \ multiplicative \
        identities: \]
        \[ \therefore \{ 0, 1=\beta^{0} \} \ \epsilon \ GF(2^{3}) \]
        \[ \therefore \beta^{1} \ \epsilon \ GF(2^{3}) \ (\because \ closure \
        of \ multiplication) \]

    \subsubsection{Automatic Symbols}

    Automatic symbols refer to terms up to $x^{n-1}$. Automatic symbols may be
    generated concurrently, and consist of the following attributes:
    \begin{enumerate}
        \item The symbols for $\{x^{0}, x^{1}, \ldots, x^{n-1}\}$ are generated
        by setting the corresponding bits to 1.
        \item The symbol for $x^{n}$ is generated by setting the corresponding
        bits for the terms in the polynomial after the highest degree term.
        \item The symbol for $x^{2^{n}-1}$ cycles back to $x^{0}$, and is set
        to $x^{0}$.
    \end{enumerate}

        Automatic symbols consist of $n+1$ terms. Therefore, the maximum of 16
        bits would have 17 terms generated by default.

   \begin{table}[h]
        \def\arraystretch{2.5}
        \caption{Automatic Symbols Generated for An Irreducible Polynomial of
        Degree $\bm{n < 16}$.}
        \centering
        \begin{tabular*}{400pt}{@{\extracolsep{\fill}} ccc}

        \textbf{Element} & \textbf{Polynomial Form} & \textbf{Symbol}
        \\
        \hline

        $\alpha^{2}$ & $0_{15} + \ldots + 0_{n-1} + \ldots +
        \alpha^{2}_{2} + 0_{1} + 0_{0}$ & $\{ 0_{15} \ldots 0_{n-1}
        \ldots 1_{2} 0_{1} 0_{0} \}$ \\

        $\ldots$ & $\ldots$ & $\ldots$ \\

        $\alpha^{n-1}$ & $0_{15} + \ldots + \alpha^{n-1}_{n-1} + \ldots
        + 0_{2} + 0_{1} + 0_{0}$ & $\{ 0_{15} \ldots 1_{n-1} \ldots 0_{2} 0_{1}
          0_{0} \}$ \\

        $\alpha^{n}$ & $0_{15} + \ldots + \alpha^{n-1}_{n-1} + \ldots
        +\alpha^{2}_{2} + \alpha^{1}_{1} + \alpha^{0}_{0}$ & $\{ 0_{15}
        \ldots x_{n-1} \ldots x_{2} x_{1} x_{0} \}$ \\

        $\alpha^{2^{n}-1}$ & $0_{15} + \ldots + 0_{n-1} + \ldots +
        \alpha^{2}_{2} + 0_{1} + 0_{0}$ & $\{ 0_{15} \ldots 0_{n-1}
        \ldots 0_{2} 0_{1} 0_{0} \}$ \\

        \end{tabular*}
        \label{table:auto_sym}
    \end{table}

    Table \ref{table:auto_sym} refers to polynomials with their highest degree
    of \bm{$n < 15$}. $0_{15}\ldots$ indicates zero- padding of the bits. For
    $\bm{n = 16}$, the most significant bit will be $1_{15}$.

        \paragraph{Analytical Approach} \leavevmode \\ Assuming the preceding
        values exist from the proof above, use the example polynomial
        \examplepoly~ to show that the $n-1$th and the $n$th elements exist in
        $GF(2^{3})$.

        \hspace*{\fill}
        \centerline{Let $\beta \ \epsilon \ GF(2^{3})$ be a root of
        \examplepoly~ $\implies \beta^{3} + \beta^{2} + \beta^{0}$}
        \hspace*{\fill}
        \centerline{$\therefore$ The coefficients are in $GF(2) \implies
        \beta^{3} = \beta^{2} + \beta^{0}$}
        \[ \therefore \beta^{2} \ \epsilon \ GF(2^{3}) \ (\because \ closure \
        of \ multiplication) \]
        \[ \therefore \beta^{3} \ \epsilon \ GF(2^{3}) \ (\because \beta^{3} =
        \beta^{2} + \beta^{0}) \]

    \subsubsection{Generated Symbols} The rest of the symbols for the elements
    $x^{n+1}$ to $x^{2^{n}-2}$ must be generated. In total, that would require
    $2^{n}-2-n-1+1=2^{n}-2-n$ terms. Therefore, the maximum of 16 bits would
    require 65,518 terms to be generated.

        \paragraph{Analytical Approach} \leavevmode \\ Assuming the preceding
        values exist from the proof above, use the example polynomial
        \examplepoly~ to show that the elements up to the $(2^{n-2})$th
        elements exist in $GF(2^{3})$.

        \hspace*{\fill}
        \centerline{Let $\beta \ \epsilon \ GF(2^{3})$ be a root of
        \examplepoly~ $\implies \beta^{3} + \beta^{2} + \beta^{0}$}
        \hspace*{\fill}
        \centerline{$\therefore$ The coefficients are in $GF(2) \implies
        \beta^{3} = \beta^{2} + \beta^{0}$}

        \begin{minipage}[t]{0.5\textwidth}
            \begin{equation*}
                \begin{split}
                    \because \beta^{4} & = \beta^{1} \times \beta^{3} \\
                    & = \beta^{1} (\beta^{2}+\beta^{0}) \\
                    & = \beta^{3}+\beta^{1} \\
                    & = \beta^{2}+\beta^{1}+\beta^{0}
                \end{split}
            \end{equation*}
            \[ \therefore \beta^{4} \ \epsilon \ GF(2^{3}) \]
        \end{minipage}
        \begin{minipage}[t]{0.5\textwidth}
            \begin{equation*}
                \begin{split}
                    \because \beta^{5} & = \beta^{1} \times \beta^{4} \\
                    & = \beta^{1} (\beta^{2}+\beta^{1}+\beta^{0}) \\
                    & = \beta^{3}+\beta^{2}+\beta^{1} \\
                    & = \beta^{2}+\beta^{0}+\beta^{2}+\beta^{1} \\
                    & = \beta^{1}+\beta^{0}
                \end{split}
            \end{equation*}
            \[ \therefore \beta^{5} \ \epsilon \ GF(2^{3}) \]
        \end{minipage}

        \begin{minipage}[t]{0.5\textwidth}
            \begin{equation*}
                \begin{split}
                    \because \beta^{6} & = \beta^{1} \times \beta^{5} \\
                    & = \beta^{1} (\beta^{1}+\beta^{0}) \\
                    & = \beta^{2}+\beta^{1}
                \end{split}
            \end{equation*}
            \[ \therefore \beta^{6} \ \epsilon \ GF(2^{3}) \]
        \end{minipage}
        \begin{minipage}[t]{0.5\textwidth}
            \begin{equation*}
                \begin{split}
                    \because \beta^{7} & = \beta^{1} \times \beta^{6} \\
                    & = \beta^{1} (\beta^{2}+\beta^{1}) \\
                    & = \beta^{3}+\beta^{2} \\
                    & = \beta^{2}+\beta^{0}+\beta^{2} \\
                    & = \beta^{0} = 1
                \end{split}
            \end{equation*}
            \[ \therefore \beta^{7} \ \epsilon \ GF(2^{3}) \]
        \end{minipage}

        \subsubsection{Example}

            \begin{table}[h]
                \def\arraystretch{1.5}
                \caption{The 8 Element Vectors of \examplepoly~ in $GF(2)[x]$}

                \centering
                \begin{tabular*}{250pt}{@{\extracolsep{\fill}} c|c|c|c}

                \textbf{Element} & \textbf{Symbol} & \textbf{Polynomial Form} &
                \textbf{Symbol} \\
                \hline
                $0$         & \footnotesize{[1]} & $0+0+0$              & 000\\
                $\beta^{0}$ & 000 & $0 + 0 + \beta^{0}$                 & 001\\
                $\beta^{1}$ & 001 & $0 + \beta^{1} + 0$                 & 010\\
                $\beta^{2}$ & 010 & $\beta^{2} + 0 + 0$                 & 100\\
                $\beta^{3}$ & 011 & $\beta^{2} + 0 + \beta^{0}$         & 101\\
                $\beta^{4}$ & 100 & $\beta^{2} + \beta^{1} + \beta^{0}$ & 111\\
                $\beta^{5}$ & 101 & $0 + \beta^{1} + \beta^{0}$         & 011\\
                $\beta^{6}$ & 110 & $\beta^{2} + \beta^{1} + 0$         & 110\\
                $\beta^{7}$ & \footnotesize{[2]} & $0 + 0 + \beta^{0}$  & 001\\
                \end{tabular*}
            \end{table}

            \footnotesize{[1]} Zero is a reserved element where its binary
            symbol does not represent its decimal value. \\
            \footnotesize{[2]} Elements beyond the $(2^{n}-1)$th element will be
            handled with special conditions since they cycle back to previous.

        \paragraph{Digital Logic} \leavevmode \\ \textbf{ELABORATE MORE}
        Generating the rest of the symbols may be implemented with a linear
        feedback shift register (LFSR), using the following recursive equation:

        \begin{equation*}
            \begin{split}
                \alpha^{n+m} & =\alpha^{n+(m-1)}\times \alpha^{n} \\
                & = (\alpha^{n+(m-1)} \ll 1 )[n-1] = 1 \Longrightarrow
                (\alpha^{n+(m-1)} \ll 1 )[n-2:0] \oplus \alpha^{n}[n-2:0] \\
                & \ \ \ \ \wedge (\alpha^{n+(m-1)} \ll 1 )[n-1] = 0
                \Longrightarrow ( \alpha^{n+(m-1)} \ll 1 )[n-2:0] \\
            \end{split}
        \end{equation*}

    \newpage

        \subsection{Operations} Operations in the Galois Field Arithmetic Unit
        consist of addition, subtraction, multiplication, division and
        logarithm.

            \subsubsection{Addition and Subtraction} Binary addition and binary subtraction
are synonymous in the Galois Field. Addition and subtraction of Galois operands
may be done by bitwise exclusive OR-ing the operands.
    \begin{equation*}
        \begin{split}
            \alpha^{i} \pm \alpha^{j} & = \{ x_{i, n}, \ldots x_{i, 2},
            x_{i, 1}, x_{i, 0} \} + \{ x_{j, n}, \ldots x_{j, 2}, x_{j, 1},
            x_{j, 0} \} \\
            & = \{(x_{i, n} \oplus x_{j,n}), \ldots (x_{i, 2} \oplus x_{j,
            2}), (x_{i, 1}\oplus x_{j, 1}), (x_{i, 0}\oplus x_{j, 0})\} \\
            & = \alpha^{k}
        \end{split}
    \end{equation*}

    \paragraph{{\small Digital Design}} \leavevmode \\ Only polynomial forms of
    inputs are valid for these operations.

    The implementation of Galois addition and subtraction may be computed with
    a single-level parallel array of XOR gates.

            \subsubsection{Multiplication} Binary multiplication of Galois operands is
congruent to the sum of the indices of the operands. If the indices sum to
greater than or equal to $2^{n}-1$, then $2^{n}-1$ is subtracted from the sum
to prevent overflow.
\begin{align*}
    \alpha^{i} \cdot \alpha^{j} & = \{ x_{i, n-1}, \ldots, x_{i, 2}, x_{i, 1},
    x_{i, 0} \} \cdot \{x_{j, n-1}, \ldots, x_{j, 2}, x_{j, 1}, x_{j, 0}\} \\
    & = \alpha^{(i + j) \ mod \ (2^{n}-1)} \\
    & = \begin{cases}
            \alpha^{(i + j) - (2^{n}-1)} & \text{if $(i + j) \geq 2^{n}-1$} \\
            \alpha^{(i + j)} & \text{if $(i + j) < 2^{n}-1$}
        \end{cases}
\end{align*}

    \paragraph{{\small Digital Design}} \leavevmode \\ Only element forms of
    inputs are valid for this operation.

    Galois multiplication requires multiple binary additions and condition
    checks. To find the product of $\alpha^{i}$ and $\alpha^{j}$,
    a carry-lookahead adder may be used to sum the elements $i+j$. The most
    significant bits of the sum and the sum $+1$ will then be OR-ed to compute
    a single-bit control signal. The control signal will be multiplexed with a
    binary $1$ to be added to the sum.
\begin{align*}
    \alpha^{i} \cdot \alpha^{j} & = \{ x_{i, n-1}, \ldots, x_{i, 2}, x_{i, 1},
    x_{i, 0} \} \cdot \{x_{j, n-1}, \ldots, x_{j, 2}, x_{j, 1}, x_{j, 0}\} \\
    & \Longrightarrow \text{Let } n = |i| = |j| \\
    % & \Longrightarrow b = i + j \\
    % & \Longrightarrow c = b[n+1] \vee (b+1)[n+1] \\
    % & \Longrightarrow d = c \wedge 1 \\
    % & \Longrightarrow k = d + b \\
    % & \Longrightarrow k = i + j + \Big(1 \ \wedge \ \big((i+j)[n+1] \ \vee \
    % (i+j+1)[n+1]\big)\Big) \\
    % & = \alpha^{k} \\
    & = \begin{cases}
            \alpha^{(i + j)} & \text{if $\Big((i+j)[n+1] \ \vee \
            (i+j+1)[n+1]\Big)=0$} \\
            \alpha^{(i + j + 1)} & \text{if $\Big((i+j)[n+1] \ \vee \
            (i+j+1)[n+1]\Big)=1$}
        \end{cases}
\end{align*}

            \subsubsection{Division} Binary division of Galois operands is congruent to the
difference of the indices of the operands. If the difference is negative, then
the absolute value of the difference is subtracted from $2^{n}-1$ to prevent
underflow. If the difference is zero, then the quotient is $\alpha^{0}$.
    \begin{align*}
        \alpha^{i} / \alpha^{j} & = \{x_{i, n-1},\ldots x_{i, 2},x_{i,
        1},x_{i, 0}\} / \{x_{j, n-1}, \ldots x_{j, 2}, x_{j, 1}, x_{j, 0}\}
        \\
        & = \alpha^{(i - j) \ mod \ (2^{n}-1)} \\
        & = \begin{cases}
            \alpha^{(2^{n}-1) - (j - i)} & \text{if $(i - j) < 0$} \\
            \alpha^{(i - j)} & \text{if $(i - j) > 0$} \\
            \alpha^{0} & \text{if $i = j, \ \{i,j\} \neq 0$} \\
            ERROR & \text{if $j = 0$}
        \end{cases}
    \end{align*}

    \paragraph{{\small Digital Design}} \leavevmode \\ Only element forms of
    inputs are valid for this operation.

\begin{align*}
    \alpha^{i} / \alpha^{j} & = \{ x_{i, n-1}, \ldots, x_{i, 2}, x_{i, 1},
    x_{i, 0} \} / \{x_{j, n-1}, \ldots, x_{j, 2}, x_{j, 1}, x_{j, 0}\} \\
    & \Longrightarrow \text{Let } n = |i| = |j| \\
    & = \begin{cases}
            \alpha^{(i + j_{2's})} & \text{if $\overline{(i+j_{2's})[n+1]}=0$} \\
            \alpha^{(i + j_{2's} + 1_{2's})} & \text{if
            $\overline{(i+j_{2's})[n+1]}=1$}
        \end{cases}
\end{align*}

            \subsubsection{Logarithm} Logarithm is considered a unary operation in the
Galois field, where only one operand is required. Logarithm here refers to the degree of the operand.
    \begin{equation*}
        log_{\alpha}(\alpha^{i}) = \deg(\alpha^{i}) = i \\
    \end{equation*}

    \paragraph{{\small Digital Design}} \leavevmode \\ Only element forms of
    inputs are valid for this operation.

    The logarithm of a Galois operand is the index of the term itself.


    \iffalse
    \section{Example}

        \newpage
        \subsection{Generate the addition (bitwise XOR) and multiplication
        tables for the implementation of $GF(2)[x]$}

            \begin{table}[h]
                \def\arraystretch{1.5}
                \caption{Addition Table for \examplepoly~ in $GF(2)[x]$}
                \centering
                \begin{tabular*}{250pt}{@{\extracolsep{\fill}} c | c c c c c c
                c c}

                    $\bm{+}$ & $\bm{0}$ & $\bm{\beta^{0}}$ & $\bm{\beta^{1}}$ &
                    $\bm{\beta^{2}}$ & $\bm{\beta^{3}}$ & $\bm{\beta^{4}}$ &
                    $\bm{\beta^{5}}$ & $\bm{\beta^{6}}$ \\
                    \hline

                    $\bm{0}$ & $0$ & $\beta^{0}$ & $\beta^{1}$ & $\beta^{2}$ &
                    $\beta^{3}$ & $\beta^{4}$ & $\beta^{5}$ & $\beta^{6}$ \\

                    $\bm{\beta^{0}}$ & $\beta^{0}$ & $0$ & $\beta^{5}$ &
                    $\beta^{3}$ & $\beta^{2}$ & $\beta^{6}$ & $\beta^{1}$ &
                    $\beta^{4}$ \\

                    $\bm{\beta^{1}}$ & $\beta^{1}$ & $\beta^{5}$ & $0$ &
                    $\beta^{6}$ & $\beta^{4}$ & $\beta^{3}$ & $\beta^{0}$ &
                    $\beta^{2}$ \\

                    $\bm{\beta^{2}}$ & $\beta^{2}$ & $\beta^{3}$ & $\beta^{6}$
                    & $0$ & $\beta^{0}$ & $\beta^{5}$ & $\beta^{4}$ &
                    $\beta^{1}$ \\

                    $\bm{\beta^{3}}$ & $\beta^{3}$ & $\beta^{2}$ & $\beta^{4}$
                    & $\beta^{0}$ & $0$ & $\beta^{1}$ & $\beta^{6}$ &
                    $\beta^{5}$ \\

                    $\bm{\beta^{4}}$ & $\beta^{4}$ & $\beta^{6}$ & $\beta^{3}$
                    & $\beta^{5}$ & $\beta^{1}$ & $0$ & $\beta^{2}$ &
                    $\beta^{0}$ \\

                    $\bm{\beta^{5}}$ & $\beta^{5}$ & $\beta^{1}$ & $\beta^{0}$
                    & $\beta^{4}$ & $\beta^{6}$ & $\beta^{2}$ & $0$ &
                    $\beta^{3}$ \\

                    $\bm{\beta^{6}}$ & $\beta^{6}$ & $\beta^{4}$ & $\beta^{2}$
                    & $\beta^{1}$ & $\beta^{5}$ & $\beta^{0}$ & $\beta^{3}$ &
                    $0$ \\

                \end{tabular*}
            \end{table}

            \begin{table}[h]
                \def\arraystretch{1.5}
                \caption{Multiplication Table for \examplepoly~ in $GF(2)[x]$}
                \centering
                \begin{tabular*}{250pt}{@{\extracolsep{\fill}} c | c c c c c c
                c c}

                    \bm{$\times$} & $\bm{0}$ & $\bm{\beta^{0}}$ &
                    $\bm{\beta^{1}}$ & $\bm{\beta^{2}}$ & $\bm{\beta^{3}}$ &
                    $\bm{\beta^{4}}$ & $\bm{\beta^{5}}$ & $\bm{\beta^{6}}$ \\
                    \hline

                    $\bm{0}$ & $0$ & $0$ & $0$ & $0$ & $0$ & $0$ & $0$ & $0$ \\

                    $\bm{\beta^{0}}$ & $0$ & $\beta^{0}$ & $\beta^{1}$ &
                    $\beta^{2}$ & $\beta^{3}$ & $\beta^{4}$ & $\beta^{5}$ &
                    $\beta^{6}$ \\

                    $\bm{\beta^{1}}$ & $0$ & $\beta^{1}$ & $\beta^{2}$ &
                    $\beta^{3}$ & $\beta^{4}$ & $\beta^{5}$ & $\beta^{6}$ &
                    $\beta^{0}$ \\

                    $\bm{\beta^{2}}$ & $0$ & $\beta^{2}$ & $\beta^{3}$ &
                    $\beta^{4}$ & $\beta^{5}$ & $\beta^{6}$ & $\beta^{0}$ &
                    $\beta^{1}$ \\

                    $\bm{\beta^{3}}$ & $0$ & $\beta^{3}$ & $\beta^{4}$ &
                    $\beta^{5}$ & $\beta^{6}$ & $\beta^{0}$ & $\beta^{1}$ &
                    $\beta^{2}$ \\

                    $\bm{\beta^{4}}$ & $0$ & $\beta^{4}$ & $\beta^{5}$ &
                    $\beta^{6}$ & $\beta^{0}$ & $\beta^{1}$ & $\beta^{2}$ &
                    $\beta^{3}$ \\

                    $\bm{\beta^{5}}$ & $0$ & $\beta^{5}$ & $\beta^{6}$ &
                    $\beta^{0}$ & $\beta^{1}$ & $\beta^{2}$ & $\beta^{3}$ &
                    $\beta^{4}$ \\

                    $\bm{\beta^{6}}$ & $0$ & $\beta^{6}$ & $\beta^{0}$ &
                    $\beta^{1}$ & $\beta^{2}$ & $\beta^{3}$ & $\beta^{4}$ &
                    $\beta^{5}$ \\

                \end{tabular*}
            \end{table}
            \fi

\end{document}
