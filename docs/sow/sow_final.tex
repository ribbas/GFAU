%%%%%%%%%%%%%%%%%%%%%%%%%%%%%%%%%%%%%%%%%%%%%%%%%%%%%%%%%%%%%%%%%%%%%%%%%%%%%%
% Template LaTeX Template Version 1.0 (December 8 2014)
%
% This template has been downloaded from: http://www.LaTeXTemplates.com
%
% Original author: Brandon Fryslie With extensive modifications by: Vel
% (vel@latextemplates.com)
%
% License: CC BY-NC-SA 3.0 (http://creativecommons.org/licenses/by-nc-sa/3.0/)
%
% Authors: Sabbir Ahmed, Jeffrey Osazuwa, Howard To, Brian Weber
% 
%%%%%%%%%%%%%%%%%%%%%%%%%%%%%%%%%%%%%%%%%%%%%%%%%%%%%%%%%%%%%%%%%%%%%%%%%%%%%%

\documentclass[12pt]{extarticle}
%%%%%%%%%%%%%%%%%%%%%%%%%%%%%%%%%%%%%%%%%
% Structure Structural Definitions File Version 1.0 (December 8 2014)
%
% Created by: Vel (vel@latextemplates.com)
% 
% This file has been downloaded from: http://www.LaTeXTemplates.com
%
% License: CC BY-NC-SA 3.0 (http://creativecommons.org/licenses/by-nc-sa/3.0/)
%
%%%%%%%%%%%%%%%%%%%%%%%%%%%%%%%%%%%%%%%%%

\usepackage{geometry}  % page layout modification

\usepackage[utf8]{inputenc}  % letters with accents
\usepackage[T1]{fontenc}  % use 8-bit encoding that has 256 glyphs
\usepackage{avant}  % Avantgarde font

\usepackage{amsmath}  % equations
\usepackage{amssymb}  % extention of symbols
\usepackage{bm}  % extention of bolding features
\usepackage{caption}  % captions for tables
\usepackage{enumitem}  % extra options for enumeration
\usepackage[pdftex]{graphicx}  % figures
\usepackage{setspace}  % extension of line spacing options

\usepackage{enumitem}

\usepackage[square,numbers]{natbib}  % for bibtex

% add more sections beyond \subsubsection
\setcounter{tocdepth}{4}
\setcounter{secnumdepth}{4}

\setlength{\parindent}{0mm} % don't indent paragraphs
\setlength{\parskip}{2.5mm} % space between paragraphs
\renewcommand{\baselinestretch}{1.5}  % space between lines

\setlength{\textwidth}{16cm} % width of the text on the page
\setlength{\textheight}{23cm} % height of the text on the page
\setlength{\oddsidemargin}{0cm} % width of the margin
\setlength{\topmargin}{-1.25cm} % reduce the top margin

\captionsetup[table]{skip=10pt}  % space between tables and their captions
\captionsetup{labelfont=bf}  % bold captions

\renewcommand\familydefault{\sfdefault}  % default font for entire document
 % specifies the document layout and style
\usepackage{tabularx}
% names
\newcommand{\team}{Galois Field Arithmetic Unit}
\newcommand{\Sabbir}{Sabbir Ahmed}
\newcommand{\Jeffrey}{Jeffrey Osazuwa}
\newcommand{\Howard}{Howard To}
\newcommand{\Brian}{Brian Weber}

% document info command
\newcommand{\documentinfo}[5]{
    \begin{centering}
        \parbox{6.8in}{
        \begin{spacing}{1}
            \begin{flushleft}
                \begin{tabular}{l l} #1 \\ #2 \\ #3 \\ #4 \\ #5 \\
                \end{tabular}\\
                \rule{\textwidth}{1pt}
            \end{flushleft}
        \end{spacing} }
    \end{centering} }

\renewcommand{\baselinestretch}{1.5}

\begin{document}

    \documentinfo {\textbf{MEMO NUMBER:} GFAU SOW} {\textbf{DATE:} {\today}}
    {\textbf{TO: } EFC LaBerge} {\textbf{FROM: }\Sabbir, \Jeffrey, \Howard,
    \Brian} {\textbf{SUBJECT: } Galois Field Arithmetic Unit Statement of Work}
    \vspace{-0.3in}

    \section{Introduction} A Galois Field is a field with a finite number of
    elements. The nomenclature $GF(q)$ is used to indicate a Galois Field with
    $q$ elements. In $GF(q)$, the parameter $q$ must be a power of a prime. For
    each prime power there exists exactly one finite field. The binary field
    $GF(2)$ is the most frequently used Galois field \cite{wolfdef}.

    The \team~will handle irreducible polynomials in $GF(2^n)$, where $2 \leq
    n \leq 16$. The arithmetic logic unit (ALU) will generate all the terms in
    the field of the polynomial, and allow the user to view and apply the
    following Galois operations: addition, subtraction, multiplication,
    division and logarithm.

        \subsection{Purpose and Scope} This Statement of Work outlines and
        elaborates the tasks necessary to implement the project. The document
        also details their corresponding milestones and deadlines and how the
        contribution will be divided within the team.

    \section{Roles and Division of Labor} This project requires equal team work
    on all tasks because of the steep learning curve on implementing
    coprocessors with programmable boards. None of the team members have
    comprehensive prior knowledge or training on field programmable gate array
    units, and are therefore required to learn the concepts concurrently.

    Although there is no clear division of labor within the team, each members
    have been implicitly designated unofficial roles. Table \ref{table:roles} provides an estimated division of labor among the members of \team~.

    \begin{table}[h]
        \renewcommand{\arraystretch}{1.5}
        \renewcommand{\baselinestretch}{1}
        \caption{Estimated Division of Labor of \team~}
        \centering

        \begin{tabularx}{\textwidth}{ | c | X | }
        \hline
        \textbf{Member} & \textbf{Responsibilities} \\
        \hline

        \textbf{Sabbir} & \noindent\parbox[c]{\hsize}{\begin{itemize}
            \item Validate the system inputs and outputs through the operations
            in the unit
            \item Provide background information on the mathematical concepts
            and theoretical design
            \item Finalize written reports and deliverables
        \end{itemize}} \\
        \hline

        \textbf{Jeffrey} & \noindent\parbox[c]{\hsize}{\begin{itemize}
            \item Design modules using the hardware description language
            \item Support the designing processes by providing test benches
            and by synthesizing the individual modules.
        \end{itemize}} \\
        \hline

        \textbf{Howard} & \noindent\parbox[c]{\hsize}{\begin{itemize}
            \item Act as the point of contact between the team and the project
            manager and other consultants
            \item Schedule of team meetings and milestones.
        \end{itemize}} \\
        \hline

        \textbf{Brian} & \noindent\parbox[c]{\hsize}{\begin{itemize}
            \item Maintain and validate the digital design of the system at
            various levels
            \item Oversee the designing of the system using the hardware
            description language
        \end{itemize}} \\
        \hline

        \end{tabularx}
        \label{table:roles}
    \end{table}

    Each members of the GFAU shall contribute to designing the individual
    modules and the entire system.

    \section{Tasks}

        \subsection{Research} In this section, we introduce a series of
        questions that shall be answered in our preliminary research relating
        to the title of each subsection.

        \subsubsection{Background} The team shall conduct extensive research on
        the mathematics and theoretical concepts behind the operations in the
        GFAU. The unit emphasizes on the terms and the unary and binary
        operations among them bounded in the Galois field. A strong
        understanding on such topics is therefore essential for successful and
        accurate computations.

        \subsubsection{Devices} The team shall conduct extensive research on
        the implementation and synthesis of digital design on programmable
        boards and their corresponding best practices.
        
            \paragraph{Field Programmable Gate Array (FPGA)} \leavevmode \\~\\
            What are designs that are commonly used, such as common arithmetic
            operations that will be helpful in the design of our system? What
            is and isn't allowed in VHDL in order for it to be synthesizable?
            What are some important specifications to look at when shopping for
            an FPGA? What are specifications do we need of our FPGA to meet our
            requirements?

            \paragraph{External Devices} \leavevmode \\~\\ What are some
            external devices that we will need to use? How will the GFAU
            communicate with these devices? What specifications are required of
            these devices to meet our system requirements?

        \subsection{Design}

            \subsubsection{System Boundary} A system boundary diagram detailing
            the functions, inputs and outputs of our system shall be developed.
            Additional diagrams elaborating on various hierarchies or the
            project shall also be developed, including the functional flow and
            data flow diagrams.

            \subsubsection{Schematics} Several schematics for the project may
            be developed and utilized during the development phase. A high-
            level schematic consisting of all the modules in the system shall
            be developed for the final product. The schematic shall be divided
            into segments to be elaborated on a lower level by individual
            members.

        \subsection{Software Implementation}

            \subsubsection{Design in VHSIC Hardware Description Language
            (VHDL)} Each of the operations in the GFAU: addition, subtraction,
            multiplication, division and logarithm, shall be implemented with
            independent and discrete VHDL modules. The source code shall be
            written and comprehensively documented using the best practices and
            standards imposed by the "IEEE Standard VHDL Language Reference
            Manual".

            \subsubsection{Simulation and Synthesis in VHDL} In order to ensure
            that the source code is functional, testing shall be performed
            after the completion of each module The modules shall be completed
            using standards for synthesis. Each modules shall be independently
            synthesized to ensure they match the intended schematics and
            designs. All simulations and software testing are expected to be
            completed by the end of November.

            \subsubsection{External Devices} External devices such as memory
            may be required in our project. Should they be needed, their
            behavior shall be abstracted in VHDL so we can integrate and test
            our code to make sure it will work with the external devices. Any
            code written to describe the behavior of external devices does not
            have to be synthesizable, however.

        \subsection{Purchases} An external memory chip suitable for storing the
        lookup tables created during the polynomial term generation shall be
        researched extensively before purchase. Research on the external memory
        interface includes simulation and testing of its truth tables using
        VHDL. A development board shall be researched on concurrently before
        being purchased for prototyping the implementation. The board shall
        comply with the constraints detailed in the System Requirement
        Specifications.

        \subsection{Hardware}

            \subsubsection{Integration} The system shall be integrated with an
            FPGA board bounded by the constraints detained in the Systems
            Requirement Specifications. The board shall successfully
            communicate with a microcontroller for user interface and its
            external memories. The hardware integration shall be completed by
            mid-semester during spring after all purchases are made final.

            \subsubsection{Hardware Testing} Extensive testing on the FPGA
            board and its external components shall be conducted before, during
            and after the integration with the VHDL system modules. The testing
            of the hardware in the system shall server as the final milestone
            for the project during the end of the spring semester.

    \newpage
    \section{Deliverables} Table 1 detail the expected deliverables throughout
    the project and shall be completed concurrently with the project
    development:

    \begin{table}[h]

        \renewcommand{\arraystretch}{1.8}
        \caption{Deliverables}
        \centering

        \begin{tabular}{ | p{5cm} | p{7cm} | l | p{5cm} |}

            \hline \textbf{Deliverable} & \textbf{Description} &
            \textbf{Deadline} \\

            \hline System Specification Requirements & A detailed specification
            of the unit shall be developed including both hardware and software
            requirements, and any of the constraints that needed to be met. The
            System Specification Requirements also detailed all the inputs and
            outputs of the GFAU. & October 18, 2017 \\

            \hline Weekly Team Status Reports & Weekly status reports by the
            GFAU team shall be submitted to Dr. LaBerge discussing the
            completed tasks and issues encountered for the current period. They
            shall also include planned tasks for the next period & Weekly \\

            \hline Preliminary Design Review & A Preliminary Design Review of
            the project shall be reviewed by Dr. LaBerge in the first week of
            December. & First week of December \\

            \hline Completed Design Review & A final review of the GFAU shall
            be reviewed by Dr. LaBerge in May, 2018. & May, 2018 \\

            \hline Demo & A fully functional GFAU shall be presented in May,
            2018 to Dr. LaBerge. & May, 2018 \\

            \hline

        \end{tabular}

    \end{table}

    \section{Timeline} Figure 1 below shows a timeline of the capstone project
    in a Gantt Chart.

    \begin{figure}[h]
        \begin{center}
            \includegraphics[width=1\textwidth]{gantt_chart.png}
            \caption{Capstone Schedule} \label{fig:gantt_chart}
        \end{center}
    \end{figure}

    \begin{thebibliography}{2}
        \bibitem{wolfdef}
        Wolfram Math World, "Finite Field." \textit{Wolfram Math World},
        2017. [Online document]. Availble:
              http://mathworld.wolfram.com/FiniteField.html. [Accessed:
              Nov. 21, 2017].

        \bibitem{crc} 
        N. Matloff , "Cyclic Redundancy Checking." \textit{University of
           California at Davis}, 2001. [Online]. Availble: http://heather.c
           s.ucdavis.edu/~matloff/Networks/CRC/Old/ErrChkCorr.html.
           [Accessed: Nov. 21, 2017].
    \end{thebibliography}

\end{document}
