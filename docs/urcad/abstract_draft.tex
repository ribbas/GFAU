\documentclass[12pt]{extarticle}
\usepackage[T1]{fontenc}
\usepackage[margin=25mm]{geometry}

\title{Galois Field Arithmetic Unit}
\author{Sabbir Ahmed, Jeffrey Osazuwa, Howard To, Brian Weber}
\date{\today}

\begin{document}

    \maketitle

    % Here is the abstract.
    \begin{abstract}

        \large
        The Galois Field Arithmetic Unit (GFAU) project was an arithmetic logic
        unit (ALU) that generated all the terms in the Galois field of an input
        primitive polynomial and allowed addition, subtraction, multiplication,
        division and logarithm operations between them. Galois Fields,
        consisting of a finite number of elements, are represented by $GF(q)$,
        where $q$ must be a power of a prime. There exists exactly one finite
        field for each prime power. The binary field is the most frequently
        used Galois Field. The GFAU will handle primitive polynomials in
        $GF(2^n)$, where $2 \le n \le 15$. Galois Fields have various
        applications in error detection and correction (EDAC). Specifically,
        cyclic redundancy checks (CRC) is an EDAC that employ $GF(2)$. EDAC has
        many expensive calculations that are difficult for low power and
        inexpensive microcontrollers to handle. The GFAU project made Galois
        Field computations more accessible to such low powered devices.

    \end{abstract}

\end{document}
