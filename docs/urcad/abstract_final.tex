\documentclass[12pt]{extarticle}
\usepackage[T1]{fontenc}
\usepackage[margin=25mm]{geometry}

\title{Galois Field Arithmetic Unit}
\author{Sabbir Ahmed, Jeffrey Osazuwa, Howard To, Brian Weber}
\date{\today}

\begin{document}

    \maketitle

    \begin{abstract}

        \large
        The Galois Field Arithmetic Unit (GFAU) generated all the elements in
        the Galois field of a binary-coded decimal primitive polynomial. The
        arithmetic logic unit (ALU) computed the degree of the inputted
        polynomial and the order of its finite field and allowed addition,
        subtraction, multiplication, division and logarithm between the
        elements. GFAU utilized a Spartan 6 FPGA for the computations, a pair
        of 512Kb asynchronous integrated circuit SRAM memory chips for storage,
        and an Arduino microcontroller for user interface. Galois fields,
        consisting of a finite number of elements, are represented by $GF(q)$,
        where $q$ must be a power of a prime. The binary field is the most
        frequently used Galois field. The GFAU handled primitive polynomials in
        $GF(2^n)$, where $2 \le n \le 15$. Galois fields have various
        applications in error detection and correction (EDAC). Specifically,
        cyclic redundancy checks (CRC) is an EDAC that employ $GF(2)$. EDAC has
        many expensive calculations that are difficult for low powered and
        inexpensive microcontrollers to handle. The GFAU project made Galois
        field computations more accessible to such low powered devices.

    \end{abstract}

\end{document}
