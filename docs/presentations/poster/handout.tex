%%%%%%%%%%%%%%%%%%%%%%%%%%%%%%%%%%%%%%%%%
% Template LaTeX Template Version 1.0 (December 8 2014)
%
% This template has been downloaded from: http://www.LaTeXTemplates.com
%
% Original author: Brandon Fryslie With extensive modifications by: Vel
% (vel@latextemplates.com)
%
% License: CC BY-NC-SA 3.0 (http://creativecommons.org/licenses/by-nc-sa/3.0/)
%
% Author: Sabbir Ahmed
%
%%%%%%%%%%%%%%%%%%%%%%%%%%%%%%%%%%%%%%%%%

\documentclass[12pt]{extarticle}
% specifies the document layout and style
%%%%%%%%%%%%%%%%%%%%%%%%%%%%%%%%%%%%%%%%%
% Structure
% Structural Definitions File
% Version 1.0 (December 8 2014)
%
% Created by:
% Vel (vel@latextemplates.com)
% 
% This file has been downloaded from:
% http://www.LaTeXTemplates.com
%
% License:
% CC BY-NC-SA 3.0 (http://creativecommons.org/licenses/by-nc-sa/3.0/)
%
%%%%%%%%%%%%%%%%%%%%%%%%%%%%%%%%%%%%%%%%%

\usepackage{geometry} % Required to modify the page layout

\usepackage{amsmath}
\usepackage{amssymb}

\usepackage[utf8]{inputenc} % Required for including letters with accents
\usepackage[T1]{fontenc} % Use 8-bit encoding that has 256 glyphs

\usepackage{avant} % Use the Avantgarde font for headings
\usepackage{setspace}

\setlength{\parindent}{0mm} % Don't indent paragraphs
\setlength{\parskip}{2.5mm} % Whitespace between paragraphs

\setlength{\textwidth}{16cm} % Width of the text on the page
\setlength{\textheight}{23cm} % Height of the text on the page
\setlength{\oddsidemargin}{0cm} % Width of the margin - negative to move text left, positive to move it right
\setlength{\topmargin}{-1.25cm} % Reduce the top margin

\renewcommand\familydefault{\sfdefault}  % default font for entire document

\newcommand{\Mod}[1]{\ (\mathrm{mod}\ #1)}
\allowdisplaybreaks

\begin{document}

    \documentinfo
    {\textbf{MEMO NAME:} GFAU URCAD Handout}
    {\textbf{SUBJECT: } Analytical Approach to Generating Elements in the Galois Field}
    {\textbf{DATE: } April 25, 2018 }

\section{Elements}

    Once a polynomial is determined irreducible and primitive, its elements may
    be generated. The number of elements grow exponentially, $2^{n}-1$, where
    $n$ is the highest degree of the polynomial. \\ This document will prove
    the generation of the elements in the polynomial
    \begin{equation*}
        GF[x](2^3) = x^3+x^2+x^0
    \end{equation*}
    as well as demonstrate sample operations between them.

        \begin{proof}

            Let $\beta \in GF(2^{3})$ be a root of $x^3+x^2+x^0$. That is, $\beta^{3}+\beta^{2}+\beta^{0}=0$\\
            $\therefore$ The coefficients are in $GF(2) \implies \beta^{3}=\beta^{2}+\beta^{0}$\\
            Since a field contains the additive and multiplicative identities,
            \begin{equation*}
                \{ 0, 1=\beta^{0} \} \in GF(2^{3})
            \end{equation*}
            Also, because of closure of multiplication in a field,
            \begin{equation*}
                \{\beta^{1}, \beta^{2}, \beta^{3}\} \in GF(2^{3})
            \end{equation*}
            But,
            \begin{equation*}
                \beta^{3} = \beta^{2} + \beta^{0}
            \end{equation*}
            \begin{minipage}[t]{0.5\textwidth}
                \begin{equation*}
                    \begin{split}
                        \because \beta^{4} & = \beta^{1} \times \beta^{3} \\
                        & = \beta^{1} \times(\beta^{2}+\beta^{0}) \\
                        & = \beta^{3}+\beta^{1} \\
                        & = \beta^{2}+\beta^{1}+\beta^{0}
                    \end{split}
                \end{equation*}
            \end{minipage}
            \begin{minipage}[t]{0.5\textwidth}
                \begin{equation*}
                    \begin{split}
                        \because \beta^{5} & = \beta^{1} \times \beta^{4} \\
                        & = \beta^{1} \times(\beta^{2}+\beta^{1}+\beta^{0}) \\
                        & = \beta^{3}+\beta^{2}+\beta^{1} \\
                        & = \beta^{2}+\beta^{0}+\beta^{2}+\beta^{1} \\
                        & = \beta^{1}+\beta^{0}
                    \end{split}
                \end{equation*}
            \end{minipage}
            \begin{minipage}[t]{0.5\textwidth}
                \begin{equation*}
                    \begin{split}
                        \because \beta^{6} & = \beta^{1} \times \beta^{5} \\
                        & = \beta^{1} \times(\beta^{1}+\beta^{0}) \\
                        & = \beta^{2}+\beta^{1}
                    \end{split}
                \end{equation*}
            \end{minipage}
            \begin{minipage}[t]{0.5\textwidth}
                \begin{equation*}
                    \begin{split}
                        \because \beta^{7} & = \beta^{1} \times \beta^{6} \\
                        & = \beta^{1} \times(\beta^{2}+\beta^{1}) \\
                        & = \beta^{3}+\beta^{2} \\
                        & = \beta^{2}+\beta^{0}+\beta^{2} \\
                        & = \beta^{0} = 1
                    \end{split}
                \end{equation*}
            \end{minipage} \\

            In conclusion, $\{0, \beta^{0}, \beta^{1}, \beta^{2}, \beta^{3}, \beta^{4}, \beta^{5}, \beta^{6}\} \in GF(2^{3})$ \qedhere

        \end{proof}

    \section{Operations} The operations supported by the Galois Field
    Arithmetic Unit are
    \begin{enumerate}[label=\textbf{(\alph*)},itemsep=1em]
        \item Addition / Subtraction (bitwise exclusive disjunction): $\beta^{i} \pm \beta^{j} = \beta^{i} \oplus \beta^{j}$
        \item Multiplication: $\beta^{i} \times \beta^{j} = \beta^{i+j \Mod{2^n-1}}$
        \item Division: $\beta^{i} \div \beta^{j} = \beta^{i-j \Mod{2^n-1}}$
        \item Logarithm: $\log\beta^{i} = i$
    \end{enumerate}

\end{document}
