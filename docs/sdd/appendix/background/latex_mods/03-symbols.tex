\subsection{Symbols}

    Once a polynomial is determined irreducible, its symbols may be generated.
    The number of terms grow exponentially, $2^{n}-1$, where $n$ is the highest
    degree of the polynomial.

    \subsubsection{Default Symbols}

    Default symbols refer to terms in the field that exist for Galois Fields of
    all irreducible polynomials of $q^n$, where $2 \leq n
    \leq 15$. Since the number of elements cannot be smaller than $2$, only
    zero and the 0th and 1st elements are shared among all fields.

    \begin{table}[h]
        \def\arraystretch{2.5}
        \caption{Default Symbols Generated for All Irreducible Polynomials}
        \centering
        \begin{tabular*}{300pt}{@{\extracolsep{\fill}} ccc}

        \textbf{Element} & \textbf{Polynomial Form} & \textbf{Symbol}
        \\
        \hline $0$ & $0_{15} + \ldots + 0_{2} + 0_{1} + 0_{0}$ & $<
        \overleftarrow{0_{15} \ \ldots \ 0_{2}} 0_{1} 0_{0} >$ \\

        $\alpha^{0}$ & $0_{15} + \ldots + 0_{2} + 0_{1} + \alpha^{0}_{0}$ &
        $<\overleftarrow{0_{15} \ \ldots \ 0_{2}} 0_{1} 1_{0} >$ \\

        $\alpha^{1}$ & $0_{15} + \ldots + 0_{2} + \alpha^{1}_{1} + 0$ &
        $<\overleftarrow{0_{15} \ \ldots \ 0_{2}} 1_{1} 0_{0} >$ \\

        \end{tabular*}
        \label{table:default_sym}
    \end{table}

    Table \ref{table:default_sym} have all the bits of their values set to $0$
    except where indicated.

        \paragraph{Analytical Approach} \leavevmode\\ Using the example
        polynomial \examplepoly~, show that 0, the 0th element and the first
        element exist in $GF(2^{3})$.

        \[ \text{Let }\beta \in GF(2^{3}) \text{ be a root of
                \examplepoly~ }\implies \beta^{3}+\beta^{2}+\beta^{0} \]
        \[ \therefore \text{The coefficients are in }GF(2) \implies
        \beta^{3}=\beta^{2}+\beta^{0} \]
        \[ \because \text{a field has additive and multiplicative identities:}
        \]
        \[ \therefore \{ 0, 1=\beta^{0} \} \in GF(2^{3}) \]
        \[ \therefore \beta^{1} \in GF(2^{3}) \ (\because \text{closure of  multiplication}) \]

    \subsubsection{Automatic Symbols}

    Automatic symbols refer to terms up to $x^{n-1}$. Automatic symbols may be
    generated concurrently, and consist of the following attributes:
    \begin{enumerate}
        \item The symbols for $\{x^{0}, x^{1}, \ldots, x^{n-1}\}$ are generated
        by setting the corresponding bits to 1.
        \item The symbol for $x^{n}$ is generated by setting the corresponding
        bits for the terms in the polynomial after the highest degree term.
        \item The symbol for $x^{2^{n}-1}$ cycles back to $x^{0}$, and is set
        to $x^{0}$.
    \end{enumerate}

        Automatic symbols consist of $n+1$ terms. Therefore, the maximum of 15
        bits would have 16 terms generated by default.

   \begin{table}[h]
        \def\arraystretch{2.5}
        \caption{Automatic Symbols Generated for An Irreducible Polynomial}
        \centering
        \begin{tabular*}{450pt}{@{\extracolsep{\fill}} ccc}

        \textbf{Element} & \textbf{Polynomial Form} & \textbf{Symbol}
        \\
        \hline

        $\alpha^{2}$ & $0_{15} + \ldots + 0_{n-1} + \ldots +
        \alpha^{2}_{2} + 0_{1} + 0_{0}$ & $< \overleftarrow{0_{15} \ \ldots \
        0_{n-1}} \ \ldots \ 1_{2} 0_{1} 0_{0} >$ \\

        $\ldots$ & $\ldots$ & $\ldots$ \\

        $\alpha^{n-1}$ & $0_{15} + \ldots + \alpha^{n-1}_{n-1} + \ldots
        + 0_{2} + 0_{1} + 0_{0}$ & $< 0_{15} \ \ldots \ 1_{n-1} \ \ldots \
          0_{2} 0_{1} 0_{0} >$ \\

        $\alpha^{n}$ & $0_{15} + \ldots + \alpha^{n-1}_{n-1} + \ldots
        +\alpha^{2}_{2} + \alpha^{1}_{1} + \alpha^{0}_{0}$ & $< 0_{15} \ 
        \ldots \ x_{n-1} \ \ldots \ x_{2} x_{1} x_{0} >$ \\

        $\alpha^{2^{n}-1}$ & $0_{15} + \ldots + 0_{n-1} + \ldots +
        \alpha^{2}_{2} + 0_{1} + 0_{0}$ & $< \overleftarrow{0_{15} \ \ldots \
        0_{2}} 0_{1} 0_{0} >$ \\

        \end{tabular*}
        \label{table:auto_sym}
    \end{table}

        \paragraph{Analytical Approach} \leavevmode \\ Assuming the preceding
        values exist from the proof above, use the example polynomial
        \examplepoly~ to show that the $n-1$th and the $n$th elements exist in
        $GF(2^{3})$.
        \[ \text{Let } \beta \in GF(2^{3}) \text{ be a root of
        \examplepoly~} \implies \beta^{3} + \beta^{2} + \beta^{0} \]
        \[ \therefore \text{The coefficients are in } GF(2) \implies
        \beta^{3} = \beta^{2} + \beta^{0} \]
        \[ \therefore \beta^{2} \in GF(2^{3}) \ (\because \text{
        closure of  multiplication}) \]
        \[ \therefore \beta^{3} \in GF(2^{3}) \ (\because \beta^{3} =
        \beta^{2} + \beta^{0}) \]

    \subsubsection{Generated Symbols} The rest of the symbols for the elements
    $x^{n+1}$ to $x^{2^{n}-2}$ must be generated. In total, that would require
    $2^{n}-2-n-1+1=2^{n}-2-n$ terms. Therefore, the maximum of 15 bits would
    require 32,751 terms to be generated.

        \paragraph{Analytical Approach} \leavevmode \\ Assuming the preceding
        values exist from the proof above, use the example polynomial
        \examplepoly~ to show that the elements up to the $(2^{n-2})$th
        elements exist in $GF(2^{3})$.
        \[ \text{Let }\beta \in GF(2^{3}) \text{ be a root of
        \examplepoly~} \implies \beta^{3} + \beta^{2} + \beta^{0} \]
        \[ \therefore \text{The coefficients are in } GF(2) \implies \beta^{3}
        = \beta^{2} + \beta^{0} \]
        \begin{minipage}[t]{0.5\textwidth}
            \begin{equation*}
                \begin{split}
                    \because \beta^{4} & = \beta^{1} \times \beta^{3} \\
                    & = \beta^{1} (\beta^{2}+\beta^{0}) \\
                    & = \beta^{3}+\beta^{1} \\
                    & = \beta^{2}+\beta^{1}+\beta^{0}
                \end{split}
            \end{equation*}
            \[ \therefore \beta^{4} \in GF(2^{3}) \]
        \end{minipage}
        \begin{minipage}[t]{0.5\textwidth}
            \begin{equation*}
                \begin{split}
                    \because \beta^{5} & = \beta^{1} \times \beta^{4} \\
                    & = \beta^{1} (\beta^{2}+\beta^{1}+\beta^{0}) \\
                    & = \beta^{3}+\beta^{2}+\beta^{1} \\
                    & = \beta^{2}+\beta^{0}+\beta^{2}+\beta^{1} \\
                    & = \beta^{1}+\beta^{0}
                \end{split}
            \end{equation*}
            \[ \therefore \beta^{5} \in GF(2^{3}) \]
        \end{minipage}

        \begin{minipage}[t]{0.5\textwidth}
            \begin{equation*}
                \begin{split}
                    \because \beta^{6} & = \beta^{1} \times \beta^{5} \\
                    & = \beta^{1} (\beta^{1}+\beta^{0}) \\
                    & = \beta^{2}+\beta^{1}
                \end{split}
            \end{equation*}
            \[ \therefore \beta^{6} \in GF(2^{3}) \]
        \end{minipage}
        \begin{minipage}[t]{0.5\textwidth}
            \begin{equation*}
                \begin{split}
                    \because \beta^{7} & = \beta^{1} \times \beta^{6} \\
                    & = \beta^{1} (\beta^{2}+\beta^{1}) \\
                    & = \beta^{3}+\beta^{2} \\
                    & = \beta^{2}+\beta^{0}+\beta^{2} \\
                    & = \beta^{0} = 1
                \end{split}
            \end{equation*}
            \[ \therefore \beta^{7} \in GF(2^{3}) \]
        \end{minipage}
        \newpage

        \paragraph{Digital Logic} \leavevmode \\ \textbf{ELABORATE MORE}
        Generating the rest of the symbols may be implemented with a linear
        feedback shift register (LFSR), using the following recursive equation:
        \begin{equation*}
            \begin{split}
                \alpha^{n+m} & =\alpha^{n+(m-1)}\times \alpha^{n} \\
                & = \begin{cases}
                        \alpha^{n+(m-1)} \ll 1 &
                        \text{if $\alpha^{n+(m-1)}[n-1] = 0$} \\
                        (\alpha^{n+(m-1)} \ll 1 ) \oplus \alpha^{n} &
                        \text{if $\alpha^{n+(m-1)}[n-1] = 1$}
                    \end{cases}
            \end{split}
        \end{equation*}
