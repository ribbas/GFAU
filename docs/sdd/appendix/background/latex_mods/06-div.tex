\subsubsection{Division} Binary division of Galois operands is congruent to the
difference of the indices of the operands. If the difference is negative, then
the absolute value of the difference is subtracted from $2^{n}-1$ to prevent
underflow. If the difference is zero, then the quotient is $\alpha^{0}$.
    \begin{align*}
        \alpha^{i} / \alpha^{j} & = \{x_{i, n-1},\ldots x_{i, 2},x_{i,
        1},x_{i, 0}\} / \{x_{j, n-1}, \ldots x_{j, 2}, x_{j, 1}, x_{j, 0}\}
        \\
        & = \alpha^{(i - j) \ mod \ (2^{n}-1)} \\
        & = \begin{cases}
            \alpha^{(2^{n}-1) - (j - i)} & \text{if $(i - j) < 0$} \\
            \alpha^{(i - j)} & \text{if $(i - j) > 0$} \\
            \alpha^{0} & \text{if $i = j, \ i \neq 0$} \\
            ERROR & \text{if $j = 0$}
        \end{cases}
    \end{align*}

    \paragraph{{\small Digital Design}} \leavevmode \\ Only element forms of
    inputs are valid for this operation.

    Galois division requires multiple binary additions and condition checks. To
    find the quotient of $\alpha^{i}$ and $\alpha^{j}$, the two's complement of
    $j$ has to first be summed with $i$. Converting to two's complement may be
    done with parallel inverters and a carry-lookahead adder. The overflow bit
    of the sum will then be used as a control signal to a multiplexer. If
    activated, the multiplexer will add the two's complement of a binary $1$ to
    the sum.
\begin{align*}
    \alpha^{i} / \alpha^{j} & = \{ x_{i, n-1}, \ldots, x_{i, 2}, x_{i, 1},
    x_{i, 0} \} / \{x_{j, n-1}, \ldots, x_{j, 2}, x_{j, 1}, x_{j, 0}\} \\
    & \Longrightarrow \text{Let } n = |i| = |j| \\
    & = \begin{cases}
            \alpha^{(i + j_{2's})} & \text{if $(i+j_{2's})[n+1]=1$} \\
            \alpha^{(i + j_{2's} + 1_{2's})} & \text{if $(i+j_{2's})[n+1]=0$}
        \end{cases}
\end{align*}
