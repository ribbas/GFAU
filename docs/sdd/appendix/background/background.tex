%%%%%%%%%%%%%%%%%%%%%%%%%%%%%%%%%%%%%%%%%
%
% Author: Sabbir Ahmed
%
%%%%%%%%%%%%%%%%%%%%%%%%%%%%%%%%%%%%%%%%%

\documentclass[11pt]{extarticle}
% specifies the document layout and style
%%%%%%%%%%%%%%%%%%%%%%%%%%%%%%%%%%%%%%%%%
% Structure
% Structural Definitions File
% Version 1.0 (December 8 2014)
%
% Created by:
% Vel (vel@latextemplates.com)
% 
% This file has been downloaded from:
% http://www.LaTeXTemplates.com
%
% License:
% CC BY-NC-SA 3.0 (http://creativecommons.org/licenses/by-nc-sa/3.0/)
%
%%%%%%%%%%%%%%%%%%%%%%%%%%%%%%%%%%%%%%%%%

\usepackage{geometry} % Required to modify the page layout

\usepackage{amsmath}
\usepackage{amssymb}

\usepackage[utf8]{inputenc} % Required for including letters with accents
\usepackage[T1]{fontenc} % Use 8-bit encoding that has 256 glyphs

\usepackage{avant} % Use the Avantgarde font for headings
\usepackage{setspace}

\setlength{\parindent}{0mm} % Don't indent paragraphs
\setlength{\parskip}{2.5mm} % Whitespace between paragraphs

\setlength{\textwidth}{16cm} % Width of the text on the page
\setlength{\textheight}{23cm} % Height of the text on the page
\setlength{\oddsidemargin}{0cm} % Width of the margin - negative to move text left, positive to move it right
\setlength{\topmargin}{-1.25cm} % Reduce the top margin

\renewcommand\familydefault{\sfdefault}  % default font for entire document


\newcommand{\Mod}[1]{\ (\mathrm{mod}\ #1)}
\newcommand{\gf}[1]{GF(#1)}
\newcommand{\examplepoly}{$x^{3}+x^{2}+x^{0}$}

\begin{document}

    \documentinfo
    {\textbf{MEMO:} GFAU System Design Document}
    {\textbf{DATE:} \today}
    {\textbf{TO:} Dr. E.F. Charles LaBerge}
    {\textbf{SUBJECT: } Mathematical Background on the Galois Field Arithmetic
    Unit}
    \vspace{-0.1in}

    % \tableofcontents
    \section{Background} A Galois Field is a field with a finite number of
elements. The nomenclature $GF(q)$ is used to indicate a Galois Field with $q$
elements. In $GF(q)$, the parameter $q$ must be a power of a prime. For each
prime power there exists exactly one finite field. The binary field $GF(2)$ is
the most frequently used Galois field. \cite{wolfdef}

    \subsection{Purpose and Scope} The Galois Field Arithmetic Unit
    operates in fields of $q^n$, where $2 \leq n \leq 16$. This document
    will demonstrate the functionality of the unit by deriving all its
    functionality through mathematical approaches. The mathematical
    algorithms will be followed by their corresponding implementations in
    digital design.

    \subsection{Terms and Keywords}

        \subsubsection{Input Primitive Polynomials} Input primitive
        polynomials in the Galois Field are represented as:

        \[ c_{n}x^{n}+\ldots+c_{2}x^{2}+c_{1}x^{1}+c_{0}x^{0}, \ where \
        c,x \ \epsilon \ GF(2)=\{0,1\} \]

        For convenience and simplicity, all the examples provided will
        refer to the following polynomial: \examplepoly~.

        \subsubsection{Elements} The elements of an input polynomial refer
        to the $2^{n}-1$ elements in the field.

        \subsubsection{Polynomial Form} The polynomial forms of the
        elements refer to the $2^{n}-1$ symbolic representations of the of
        an input primitive polynomials in the field.

        \subsubsection{Example} An example of the elements and their
        corresponding polynomials is provided below:

            \begin{table}[h]
                \def\arraystretch{1.5}
                \caption{The 8 Element Vectors of \examplepoly~ in $GF(2)[x]$}

                \centering
                \begin{tabular*}{250pt}{@{\extracolsep{\fill}} c|c|c|c}

                \textbf{Element} & \textbf{Symbol} & \textbf{Polynomial Form} &
                \textbf{Symbol} \\
                \hline
                $0$         & {\scriptsize [1]} & $0+0+0$               & 000\\
                $\beta^{0}$ & 000 & $0 + 0 + \beta^{0}$                 & 001\\
                $\beta^{1}$ & 001 & $0 + \beta^{1} + 0$                 & 010\\
                $\beta^{2}$ & 010 & $\beta^{2} + 0 + 0$                 & 100\\
                $\beta^{3}$ & 011 & $\beta^{2} + 0 + \beta^{0}$         & 101\\
                $\beta^{4}$ & 100 & $\beta^{2} + \beta^{1} + \beta^{0}$ & 111\\
                $\beta^{5}$ & 101 & $0 + \beta^{1} + \beta^{0}$         & 011\\
                $\beta^{6}$ & 110 & $\beta^{2} + \beta^{1} + 0$         & 110\\
                $\beta^{7}$ & {\scriptsize [2]} & $0 + 0 + \beta^{0}$   & 001\\
                \end{tabular*}
            \end{table}

            {\scriptsize [1]} Zero is a reserved element where its binary
            symbol does not represent its decimal value. \\ {\scriptsize [2]}
            Elements beyond the $(2^{n}-1)$th element will be handled with
            special conditions since they cycle back to previous elements.

        \subsubsection{Notations} Several notations are used in this document
        to assist in linking the mathematical and digital design concepts. A
        list of selected notations are provided below:

        \begin{table}[h]
            \def\arraystretch{2}
            \caption{Mathematical and Logical Notations}

            \centering
            \begin{tabular*}{470pt}{@{\extracolsep{\fill}} r p{9cm}}

            \textbf{Notation} & \textbf{Definition} \\
            \hline
            $a \ + \ b$ & Arithmetic Addition of $a$ and $b$\\
            $a \ - \ b$ & Arithmetic Subtraction of $a$ and $b$\\
            $a \ \cdot \ b$ & Arithmetic Multiplication of $a$ and $b$\\
            $a \ / \ b$ & Arithmetic Division of $a$ and $b$\\
            $a \ \wedge \ b$ & Logical AND of $a$ and $b$\\
            $a \ \vee \ b$ & Logical OR of $a$ and $b$\\
            $a \ \oplus \ b$ & Logical XOR of $a$ and $b$\\
            $a \ \ll \ b$ & Logical Shift-Left $a$ by $b$ Bits\\
            $\overline{a}$ & Bitwise Inversion of $a$\\
            $a_{2's}$ & Two's Complement of $a$\\
            $|a|$ & Number of Bits in $a$\\
            $<x_{n}, \ x_{n-1}, \ \ldots \ x_{0}>$ & Ordered Array of Size $n$
            \\
            $\{x_{n}, \ x_{n-1}, \ \ldots \ x_{0}\}$ & Unordered Set of Size
            $n$ \\
            $< x_{m}, \ \ \overleftarrow{0_{m-1}, \ \ldots \ 0_{n}}, x_{n-1}, \
            \ldots \ x_{0} >$ & Zero-Padding Between the $(m-1)$th and $n$th
            Bits \\
            $A[i]$ & $i$th Bit of Array $A$ \\
            $A[i:j]$ & Subarray of Array $A$ from its $i$th to $j$th index
            inclusive; where $i > j$ \\

            \end{tabular*}
        \end{table}


    \newpage


    \section{Design} Input polynomials will be represented as $n$-bit
    binary-coded decimal (BCD) arrays.

    For example, the polynomial \examplepoly~ will be represented in a system
    with 16-bit data words as
        \begin{equation*}
            \begin{split}
                <0000 \ 0000 \ 0000 \ 1101> & \text{  (3rd, 2nd, and 0th bits)}
            \end{split}
        \end{equation*}

        \subsection{Primitive Polynomial} A polynomial is said to be irreducible if and
only if there exists no roots for it. A primitive polynomial is an irreducible
polynomial that generates all elements of an extension field from a base field.
\cite{primitive}


        \subsection{Symbols}

    Once a polynomial is determined irreducible, its symbols may be generated.
    The number of terms grow exponentially, $2^{n}-1$, where $n$ is the highest
    degree of the polynomial.

    \subsubsection{Default Symbols}

    Default symbols refer to terms in the field that exist for Galois Fields of
    all irreducible polynomials of $q^n$, where $2 \leq n
    \leq 15$. Since the number of elements cannot be smaller than $2$, only
    zero and the 0th and 1st elements are shared among all fields.

    \begin{table}[h]
        \def\arraystretch{2.5}
        \caption{Default Symbols Generated for All Irreducible Polynomials}
        \centering
        \begin{tabular*}{300pt}{@{\extracolsep{\fill}} ccc}

        \textbf{Element} & \textbf{Polynomial Form} & \textbf{Symbol}
        \\
        \hline $0$ & $0_{15} + \ldots + 0_{2} + 0_{1} + 0_{0}$ & $<
        \overleftarrow{0_{15} \ \ldots \ 0_{2}} 0_{1} 0_{0} >$ \\

        $\alpha^{0}$ & $0_{15} + \ldots + 0_{2} + 0_{1} + \alpha^{0}_{0}$ &
        $<\overleftarrow{0_{15} \ \ldots \ 0_{2}} 0_{1} 1_{0} >$ \\

        $\alpha^{1}$ & $0_{15} + \ldots + 0_{2} + \alpha^{1}_{1} + 0$ &
        $<\overleftarrow{0_{15} \ \ldots \ 0_{2}} 1_{1} 0_{0} >$ \\

        \end{tabular*}
        \label{table:default_sym}
    \end{table}

    Table \ref{table:default_sym} have all the bits of their values set to $0$
    except where indicated.

        \paragraph{Analytical Approach} \leavevmode\\ Using the example
        polynomial \examplepoly~, show that 0, the 0th element and the first
        element exist in $GF(2^{3})$.

        \[ \text{Let }\beta \in GF(2^{3}) \text{ be a root of
                \examplepoly~ }\implies \beta^{3}+\beta^{2}+\beta^{0} \]
        \[ \therefore \text{The coefficients are in }GF(2) \implies
        \beta^{3}=\beta^{2}+\beta^{0} \]
        \[ \because \text{a field has additive and multiplicative identities:}
        \]
        \[ \therefore \{ 0, 1=\beta^{0} \} \in GF(2^{3}) \]
        \[ \therefore \beta^{1} \in GF(2^{3}) \ (\because \text{closure of  multiplication}) \]

    \subsubsection{Automatic Symbols}

    Automatic symbols refer to terms up to $x^{n-1}$. Automatic symbols may be
    generated concurrently, and consist of the following attributes:
    \begin{enumerate}
        \item The symbols for $\{x^{0}, x^{1}, \ldots, x^{n-1}\}$ are generated
        by setting the corresponding bits to 1.
        \item The symbol for $x^{n}$ is generated by setting the corresponding
        bits for the terms in the polynomial after the highest degree term.
        \item The symbol for $x^{2^{n}-1}$ cycles back to $x^{0}$, and is set
        to $x^{0}$.
    \end{enumerate}

        Automatic symbols consist of $n+1$ terms. Therefore, the maximum of 15
        bits would have 16 terms generated by default.

   \begin{table}[h]
        \def\arraystretch{2.5}
        \caption{Automatic Symbols Generated for An Irreducible Polynomial}
        \centering
        \begin{tabular*}{450pt}{@{\extracolsep{\fill}} ccc}

        \textbf{Element} & \textbf{Polynomial Form} & \textbf{Symbol}
        \\
        \hline

        $\alpha^{2}$ & $0_{15} + \ldots + 0_{n-1} + \ldots +
        \alpha^{2}_{2} + 0_{1} + 0_{0}$ & $< \overleftarrow{0_{15} \ \ldots \
        0_{n-1}} \ \ldots \ 1_{2} 0_{1} 0_{0} >$ \\

        $\ldots$ & $\ldots$ & $\ldots$ \\

        $\alpha^{n-1}$ & $0_{15} + \ldots + \alpha^{n-1}_{n-1} + \ldots
        + 0_{2} + 0_{1} + 0_{0}$ & $< 0_{15} \ \ldots \ 1_{n-1} \ \ldots \
          0_{2} 0_{1} 0_{0} >$ \\

        $\alpha^{n}$ & $0_{15} + \ldots + \alpha^{n-1}_{n-1} + \ldots
        +\alpha^{2}_{2} + \alpha^{1}_{1} + \alpha^{0}_{0}$ & $< 0_{15} \ 
        \ldots \ x_{n-1} \ \ldots \ x_{2} x_{1} x_{0} >$ \\

        $\alpha^{2^{n}-1}$ & $0_{15} + \ldots + 0_{n-1} + \ldots +
        \alpha^{2}_{2} + 0_{1} + 0_{0}$ & $< \overleftarrow{0_{15} \ \ldots \
        0_{2}} 0_{1} 0_{0} >$ \\

        \end{tabular*}
        \label{table:auto_sym}
    \end{table}

        \paragraph{Analytical Approach} \leavevmode \\ Assuming the preceding
        values exist from the proof above, use the example polynomial
        \examplepoly~ to show that the $n-1$th and the $n$th elements exist in
        $GF(2^{3})$.
        \[ \text{Let } \beta \in GF(2^{3}) \text{ be a root of
        \examplepoly~} \implies \beta^{3} + \beta^{2} + \beta^{0} \]
        \[ \therefore \text{The coefficients are in } GF(2) \implies
        \beta^{3} = \beta^{2} + \beta^{0} \]
        \[ \therefore \beta^{2} \in GF(2^{3}) \ (\because \text{
        closure of  multiplication}) \]
        \[ \therefore \beta^{3} \in GF(2^{3}) \ (\because \beta^{3} =
        \beta^{2} + \beta^{0}) \]

    \subsubsection{Generated Symbols} The rest of the symbols for the elements
    $x^{n+1}$ to $x^{2^{n}-2}$ must be generated. In total, that would require
    $2^{n}-2-n-1+1=2^{n}-2-n$ terms. Therefore, the maximum of 15 bits would
    require 32,751 terms to be generated.

        \paragraph{Analytical Approach} \leavevmode \\ Assuming the preceding
        values exist from the proof above, use the example polynomial
        \examplepoly~ to show that the elements up to the $(2^{n-2})$th
        elements exist in $GF(2^{3})$.
        \[ \text{Let }\beta \in GF(2^{3}) \text{ be a root of
        \examplepoly~} \implies \beta^{3} + \beta^{2} + \beta^{0} \]
        \[ \therefore \text{The coefficients are in } GF(2) \implies \beta^{3}
        = \beta^{2} + \beta^{0} \]
        \begin{minipage}[t]{0.5\textwidth}
            \begin{equation*}
                \begin{split}
                    \because \beta^{4} & = \beta^{1} \times \beta^{3} \\
                    & = \beta^{1} (\beta^{2}+\beta^{0}) \\
                    & = \beta^{3}+\beta^{1} \\
                    & = \beta^{2}+\beta^{1}+\beta^{0}
                \end{split}
            \end{equation*}
            \[ \therefore \beta^{4} \in GF(2^{3}) \]
        \end{minipage}
        \begin{minipage}[t]{0.5\textwidth}
            \begin{equation*}
                \begin{split}
                    \because \beta^{5} & = \beta^{1} \times \beta^{4} \\
                    & = \beta^{1} (\beta^{2}+\beta^{1}+\beta^{0}) \\
                    & = \beta^{3}+\beta^{2}+\beta^{1} \\
                    & = \beta^{2}+\beta^{0}+\beta^{2}+\beta^{1} \\
                    & = \beta^{1}+\beta^{0}
                \end{split}
            \end{equation*}
            \[ \therefore \beta^{5} \in GF(2^{3}) \]
        \end{minipage}

        \begin{minipage}[t]{0.5\textwidth}
            \begin{equation*}
                \begin{split}
                    \because \beta^{6} & = \beta^{1} \times \beta^{5} \\
                    & = \beta^{1} (\beta^{1}+\beta^{0}) \\
                    & = \beta^{2}+\beta^{1}
                \end{split}
            \end{equation*}
            \[ \therefore \beta^{6} \in GF(2^{3}) \]
        \end{minipage}
        \begin{minipage}[t]{0.5\textwidth}
            \begin{equation*}
                \begin{split}
                    \because \beta^{7} & = \beta^{1} \times \beta^{6} \\
                    & = \beta^{1} (\beta^{2}+\beta^{1}) \\
                    & = \beta^{3}+\beta^{2} \\
                    & = \beta^{2}+\beta^{0}+\beta^{2} \\
                    & = \beta^{0} = 1
                \end{split}
            \end{equation*}
            \[ \therefore \beta^{7} \in GF(2^{3}) \]
        \end{minipage}
        \newpage

        \paragraph{Digital Logic} \leavevmode \\ \textbf{ELABORATE MORE}
        Generating the rest of the symbols may be implemented with a linear
        feedback shift register (LFSR), using the following recursive equation:
        \begin{equation*}
            \begin{split}
                \alpha^{n+m} & =\alpha^{n+(m-1)}\times \alpha^{n} \\
                & = \begin{cases}
                        (\alpha^{n+(m-1)} \ll 1 )[n-2:0] & \text{if
                        $(\alpha^{n+(m-1)} \ll 1 )[n-1] = 0$} \\
                        (\alpha^{n+(m-1)} \ll 1 )[n-2:0] \oplus
                        \alpha^{n}[n-2:0] & \text{if $(\alpha^{n+(m-1)} \ll 1
                        )[n-1] = 1$}
                    \end{cases}
            \end{split}
        \end{equation*}

        \subsection{Operations} GFAU supports addition, subtraction,
        multiplication, division and logarithm of elements in Galois fields.
        The \texttt{operators} module consists of the \texttt{addsub},
        \texttt{mul}, and \texttt{div} modules that are responsible for their
        corresponding binary operations. Additional helper module,
        \texttt{maskedtwoscmp}, is included to compute the two's complement of
        the second operand required for division. The other two modules,
        \texttt{outselect} and \texttt{outconvert} act as advanced multiplexers
        to determine the output of the operation requested.

        \subsubsection{Addition and Subtraction} Binary addition and binary subtraction
are synonymous in the Galois Field. Addition and subtraction of Galois operands
may be done by computing the bitwise exclusive disjunction of the operands.
    \begin{equation*}
        \begin{split}
            \alpha^{i} \pm \alpha^{j} & = \{ x_{i, n}, \ldots x_{i, 2},
            x_{i, 1}, x_{i, 0} \} \pm \{ x_{j, n}, \ldots x_{j, 2}, x_{j, 1},
            x_{j, 0} \} \\
            & = \{(x_{i, n} \oplus x_{j,n}), \ldots, (x_{i, 2} \oplus x_{j,
            2}), (x_{i, 1}\oplus x_{j, 1}), (x_{i, 0}\oplus x_{j, 0})\} \\
            & = \alpha^{k}
        \end{split}
    \end{equation*}

    \paragraph{{\small Digital Design}} \leavevmode \\ Only polynomial forms of
    inputs are valid for these operations.

    The implementation of Galois addition and subtraction may be computed with
    a single-level parallel array of XOR gates.

        \subsubsection{Multiplication} Binary multiplication of Galois operands is
congruent to the sum of the indices of the operands. If the indices sum to
greater than or equal to $2^{n}-1$, then $2^{n}-1$ is subtracted from the sum
to prevent overflow.
\begin{align*}
    \alpha^{i} \cdot \alpha^{j} & = \{ x_{i, n-1}, \ldots, x_{i, 2}, x_{i, 1},
    x_{i, 0} \} \cdot \{x_{j, n-1}, \ldots, x_{j, 2}, x_{j, 1}, x_{j, 0}\} \\
    & = \alpha^{(i + j) \Mod{(2^{n}-1)}} \\
    & = \begin{cases}
            \alpha^{(i + j) - (2^{n}-1)} & \text{if $(i + j) \geq 2^{n}-1$} \\
            \alpha^{(i + j)} & \text{if $(i + j) < 2^{n}-1$}
        \end{cases}
\end{align*}

    \paragraph{{\small Digital Design}} \leavevmode \\ Only element forms of
    inputs are valid for this operation.

    Galois multiplication requires multiple binary additions and condition
    checks. To find the product of $\alpha^{i}$ and $\alpha^{j}$,
    a carry-lookahead adder may be used to sum the elements $i+j$. The most
    significant bits of the sum and the sum $+1$ will then be OR-ed to compute
    a single-bit control signal. The control signal will be multiplexed with a
    binary $1$ to be added to the sum.
\begin{align*}
    \alpha^{i} \cdot \alpha^{j} & = \{ x_{i, n-1}, \ldots, x_{i, 2}, x_{i, 1},
    x_{i, 0} \} \cdot \{x_{j, n-1}, \ldots, x_{j, 2}, x_{j, 1}, x_{j, 0}\} \\
    & = \begin{cases}
            \alpha^{(i + j)} & \text{if $\Big((i+j)[n] \ \vee \
            (i+j+1)[n]\Big)=0$} \\
            \alpha^{(i + j + 1)} & \text{if $\Big((i+j)[n] \ \vee \
            (i+j+1)[n]\Big)=1$}
        \end{cases}
\end{align*}

        \subsubsection{Division} Binary division of Galois operands is congruent to the
difference of the indices of the operands. If the difference is negative, then
the absolute value of the difference is subtracted from $2^{n}-1$ to prevent
underflow. If the difference is zero, then the quotient is $\alpha^{0}$.
    \begin{align*}
        \alpha^{i} / \alpha^{j} & = \{x_{i, n-1},\ldots x_{i, 2},x_{i,
        1},x_{i, 0}\} / \{x_{j, n-1}, \ldots x_{j, 2}, x_{j, 1}, x_{j, 0}\}
        \\
        & = \alpha^{(i - j) \ mod \ (2^{n}-1)} \\
        & = \begin{cases}
            \alpha^{(2^{n}-1) - (j - i)} & \text{if $(i - j) < 0$} \\
            \alpha^{(i - j)} & \text{if $(i - j) > 0$} \\
            \alpha^{0} & \text{if $i = j, \ i \neq 0$} \\
            ERROR & \text{if $j = 0$}
        \end{cases}
    \end{align*}

    \paragraph{{\small Digital Design}} \leavevmode \\ Only element forms of
    inputs are valid for this operation.

    Galois division requires multiple binary additions and condition checks. To
    find the quotient of $\alpha^{i}$ and $\alpha^{j}$, the two's complement of
    $j$ has to first be summed with $i$. Converting to two's complement may be
    done with parallel inverters and a carry-lookahead adder. The most
    significant bit of the sum will then be used as a control signal to a
    multiplexer. If activated, the multiplexer will add the two's complement of
    a binary $1$ to the sum.
\begin{align*}
    \alpha^{i} / \alpha^{j} & = \{ x_{i, n-1}, \ldots, x_{i, 2}, x_{i, 1},
    x_{i, 0} \} / \{x_{j, n-1}, \ldots, x_{j, 2}, x_{j, 1}, x_{j, 0}\} \\
    & \Longrightarrow \text{Let } n = |i| = |j| \\
    & = \begin{cases}
            \alpha^{(i + j_{2's})} & \text{if $\overline{(i+j_{2's})[n+1]}=0$} \\
            \alpha^{(i + j_{2's} + 1_{2's})} & \text{if
            $\overline{(i+j_{2's})[n+1]}=1$}
        \end{cases}
\end{align*}

        \subsubsection{Logarithm} Logarithm is considered a unary operation in the
Galois Field, where only one operand is required. The base of the logarithm
operation in the Galois Field is implicitly 2. The logarithm of a Galois
operand is congruent to its index.
    \begin{equation*}
        \begin{split}
            log_{GF(2)}(\alpha^{i}) & = i \\
        \end{split}
    \end{equation*}

    \paragraph{{\small Digital Design}} \leavevmode \\ Only element forms of
    inputs are valid for this operation.


        \subsection{Arithmetic Exceptions}

        \subsubsection{Zero} Zero, or the null symbol, is the numerical zero used to
fulfill its role as the additive identity in the Galois field. The following
operations demonstrate the characteristics of zero in the Galois field.

    \begin{table}[h]
        \def\arraystretch{1.5}
        \caption{Operations with $\bm{0}$ in $GF(2)[x]$}

        \centering
        \begin{tabular*}{250pt}{@{\extracolsep{\fill}} c|c|c|c}

        \textbf{Operand 1} & \textbf{Operation} & \textbf{Operand 2} &
        \textbf{Output} \\
        \hline
        $\bm{0}$    & $+/-$             & $\beta^{x}$   & $\beta^{x}$ \\
        $\beta^{x}$ & $+/-$             & $\bm{0}$      & $\beta^{x}$ \\
        $\bm{0}$    & $\times$          & $\beta^{x}$   & $\bm{0}$ \\
        $\beta^{x}$ & $\times$          & $\bm{0}$      & $\bm{0}$ \\
        $\bm{0}$    & $\div$            & $\beta^{x}$   & $\bm{0}$ \\
        $\beta^{x}$ & $\div$            & $\bm{0}$      & ERROR \\
        $\bm{0}$    & $\log_{GF(2)}$    & DON'T CARE    & ERROR \\
        \end{tabular*}
    \end{table}

    \paragraph{{\small Digital Design}} \leavevmode \\ To handle zero,
    multiplexers are used to decode the intended operation to determine the
    forms if the inputs. Once the form is determined, the module raises a
    \texttt{zero flag} which is handled by individual operations to raise
    exceptions.
\begin{align*}
    \texttt{zero flag} & = \begin{cases}
        \text{$\wedge$\big(operand\big),} & \text{if operation $\in$
        \{addition, subtraction\} } \\
        \text{$\vee$\big(operand\big),} & \text{if operation $\in$
        \{multiplication, division, logarithm\} } \\
    \end{cases}
\end{align*}
\newpage

        \subsubsection{Upper Bound} 

        \bibliographystyle{abbrv}
        \bibliography{sdd/appendix/background/finitefield,sdd/appendix/background/primitive}

\end{document}
