\subsubsection{Zero} Zero, or the null symbol, is the numerical zero used to
fulfill its role as the additive identity in the Galois field. The following
operations demonstrate the characteristics of zero in the Galois field.

    \begin{table}[h]
        \def\arraystretch{1.5}
        \caption{Operations with $\bm{0}$ in $GF(2)[x]$}

        \centering
        \begin{tabular*}{250pt}{@{\extracolsep{\fill}} c|c|c|c}

        \textbf{Operand 1} & \textbf{Operation} & \textbf{Operand 2} &
        \textbf{Output} \\
        \hline
        $\bm{0}$    & $+/-$     & $\beta^{x}$   & $\beta^{x}$ \\
        $\beta^{x}$ & $+/-$     & $\bm{0}$      & $\beta^{x}$ \\
        $\bm{0}$    & $\times$  & $\beta^{x}$   & $\bm{0}$ \\
        $\beta^{x}$ & $\times$  & $\bm{0}$      & $\bm{0}$ \\
        $\bm{0}$    & $\div$    & $\beta^{x}$   & $\bm{0}$ \\
        $\beta^{x}$ & $\div$    & $\bm{0}$      & $ERROR$ \\
        $\bm{0}$    & $\deg$    & $DON'T CARE$  & $ERROR$ \\
        \end{tabular*}
    \end{table}

    \paragraph{{\small Digital Design}} \leavevmode \\ To handle zero,
    multiplexers are used to decode the intended operation to determine the
    forms if the inputs. Once the form is determined, the module raises a
    \texttt{zero flag} which is handled by individual operations to raise
    exceptions.

    \begin{table}[h]
        \def\arraystretch{1.7}
        \caption{Operations with $\bm{0}$ in $GF(2)[x]$}

        \centering
        \begin{tabular*}{250pt}{@{\extracolsep{\fill}} c|c|c}

        \textbf{Memory Block} & \textbf{Address} & \textbf{Data} \\
        \hline
        $\bm{0}$    & $< \overleftarrow{1_{m-1} \ \ldots \ 1_{2}} 1_{1} 1_{0}>$   & $< \overleftarrow{0_{m-1} \ \ldots \ 0_{2}} 0_{1} 0_{0}>$ \\
        $\bm{1}$    & $< \overleftarrow{0_{m-1} \ \ldots \ 0_{2}} 0_{1} 0_{0}>$      & $< \overleftarrow{1_{m-1} \ \ldots \ 1_{2}} 1_{1} 1_{0}>$ \\
        \end{tabular*}
    \end{table}


% \begin{align*}
%     \texttt{zero flag} & = \begin{cases}
%         \text{$\wedge$\big(operand\big),} & \text{if operation $\in$
%         \{addition, subtraction\} } \\
%         \text{$\vee$\big(operand\big),} & \text{if operation $\in$
%         \{multiplication, division, logarithm\} } \\
%     \end{cases}
% \end{align*}
% \newpage
