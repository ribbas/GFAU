\subsubsection{Multiplication} Binary multiplication of Galois operands is
congruent to the sum of the indices of the operands. If the indices sum to
greater than or equal to $2^{n}-1$, then $2^{n}-1$ is subtracted from the sum
to prevent overflow.
\begin{align*}
    \alpha^{i} \cdot \alpha^{j} & = \{ x_{i, n-1}, \ldots, x_{i, 2}, x_{i, 1},
    x_{i, 0} \} \cdot \{x_{j, n-1}, \ldots, x_{j, 2}, x_{j, 1}, x_{j, 0}\} \\
    & = \alpha^{(i + j) \Mod{(2^{n}-1)}} \\
    & = \begin{cases}
            \alpha^{(i + j) - (2^{n}-1)} & \text{if $(i + j) \geq 2^{n}-1$} \\
            \alpha^{(i + j)} & \text{if $(i + j) < 2^{n}-1$}
        \end{cases}
\end{align*}

    \paragraph{{\small Modules}} \leavevmode \\ Only element forms of
    inputs are valid for this operation.

    Galois multiplication requires multiple binary additions and condition
    checks. To find the product of $\alpha^{i}$ and $\alpha^{j}$, an adder may
    be used to sum the elements $i+j$. The most significant bits of the sum and
    the sum $+1$ will then be OR-ed to compute a single-bit control signal. The
    control signal will be multiplexed with a binary $1$ to be added to the
    sum.
\begin{align*}
    \alpha^{i} \cdot \alpha^{j} & = \{ x_{i, n-1}, \ldots, x_{i, 2}, x_{i, 1},
    x_{i, 0} \} \cdot \{x_{j, n-1}, \ldots, x_{j, 2}, x_{j, 1}, x_{j, 0}\} \\
    & = \begin{cases}
            \alpha^{(i + j)} & \text{if $\Big((i+j)[n] \ \vee \
            (i+j+1)[n]\Big)=0$} \\
            \alpha^{(i + j + 1)} & \text{if $\Big((i+j)[n] \ \vee \
            (i+j+1)[n]\Big)=1$}
        \end{cases}
\end{align*}
