%
% Originally created by: Vel (vel@latextemplates.com)
% This file has been downloaded from: http://www.LaTeXTemplates.com
% License: CC BY-NC-SA 3.0 (http://creativecommons.org/licenses/by-nc-sa/3.0/)
%
% Extinsively modified to accomodate current documents
%

\usepackage{geometry} % Required to modify the page layout

\usepackage{amsmath}
\usepackage{amssymb}
\usepackage{amsthm}
\usepackage{avant} % Use the Avantgarde font for headings
\usepackage{bm}
\usepackage{caption}
\usepackage[pdftex]{graphicx}
\usepackage{enumitem}
\usepackage{mathtools}

\usepackage[utf8]{inputenc} % Required for including letters with accents
\usepackage[T1]{fontenc} % Use 8-bit encoding that has 256 glyphs
\usepackage{mathptmx}
\usepackage{setspace}
\usepackage[square,numbers]{natbib}  % for bibtex

% add more sections beyond \subsubsection
\setcounter{tocdepth}{4}
\setcounter{secnumdepth}{4}

% margins
\setlength{\parindent}{0mm} % don't indent paragraphs
\setlength{\parskip}{2mm} % space between paragraphs

\setlength{\textwidth}{16cm} % width of the text on the page
\setlength{\textheight}{23cm} % height of the text on the page
\setlength{\oddsidemargin}{0cm} % width of the margin
\setlength{\topmargin}{-2cm} % reduce the top margin

% increase gaps in tables and their captions
\renewcommand{\arraystretch}{1.5}
\renewcommand{\baselinestretch}{1.5}  % space between lines

% table captions
\captionsetup[table]{skip=10pt}  % space between tables and their captions
\captionsetup{labelfont=bf}  % bold captions

% bold item numbers for enumerate and itemize
\setlist[enumerate,1]{label=\textbf{(\alph*)}.,itemsep=2em}
\setlist[enumerate,2]{label=\textbf{(\alph*)},itemsep=2em}
\setlist[itemize]{label=\textbf{\alph*}.,itemsep=2em}

\newcommand{\salign}[1]{\begingroup\addtolength{\jot}{#1em}}
\newenvironment{cproof}{\begin{proof}[\unskip\nopunct]}{\end{proof}}
\newcommand*\diff{\mathop{}\!\mathrm{d}}

% custom command to make headings easier
\newcommand{\documentinfo}[4]{
    \begin{centering}
        \parbox{2in}{
        \begin{spacing}{1}
            \begin{flushleft}
                \begin{tabular}{l l} #1 \\ #2 \\ #3 \\
                \end{tabular} \\
                \rule{\textwidth}{1pt}
            \end{flushleft}
        \end{spacing} }
    \end{centering} }
