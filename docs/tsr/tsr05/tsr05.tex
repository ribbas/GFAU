%%%%%%%%%%%%%%%%%%%%%%%%%%%%%%%%%%%%%%%%%
% Contract LaTeX Template Version 1.0 (December 8 2014)
%
% This template has been downloaded from: http://www.LaTeXTemplates.com
%
% Original author: Brandon Fryslie With extensive modifications by: Vel
% (vel@latextemplates.com)
%
% License: CC BY-NC-SA 3.0 (http://creativecommons.org/licenses/by-nc-sa/3.0/)
%
% Authors: Sabbir Ahmed, Jeffrey Osazuwa, Howard To, Brian Weber
% 
%%%%%%%%%%%%%%%%%%%%%%%%%%%%%%%%%%%%%%%%%

\documentclass[paper=usletter, fontsize=12pt]{article}
%%%%%%%%%%%%%%%%%%%%%%%%%%%%%%%%%%%%%%%%%
% Structure
% Structural Definitions File
% Version 1.0 (December 8 2014)
%
% Created by:
% Vel (vel@latextemplates.com)
% 
% This file has been downloaded from:
% http://www.LaTeXTemplates.com
%
% License:
% CC BY-NC-SA 3.0 (http://creativecommons.org/licenses/by-nc-sa/3.0/)
%
%%%%%%%%%%%%%%%%%%%%%%%%%%%%%%%%%%%%%%%%%

\usepackage{geometry} % Required to modify the page layout

\usepackage{amsmath}
\usepackage{amssymb}

\usepackage[utf8]{inputenc} % Required for including letters with accents
\usepackage[T1]{fontenc} % Use 8-bit encoding that has 256 glyphs

\usepackage{avant} % Use the Avantgarde font for headings
\usepackage{setspace}

\setlength{\parindent}{0mm} % Don't indent paragraphs
\setlength{\parskip}{2.5mm} % Whitespace between paragraphs

\setlength{\textwidth}{16cm} % Width of the text on the page
\setlength{\textheight}{23cm} % Height of the text on the page
\setlength{\oddsidemargin}{0cm} % Width of the margin - negative to move text left, positive to move it right
\setlength{\topmargin}{-1.25cm} % Reduce the top margin

\renewcommand\familydefault{\sfdefault}  % default font for entire document
 % specifies the document layout and style

\newcommand{\team}{Galois Field Arithmetic Unit}
\newcommand{\Sabbir}{Sabbir Ahmed}
\newcommand{\Jeffrey}{Jeffrey Osazuwa}
\newcommand{\Howard}{Howard To}
\newcommand{\Brian}{Brian Weber}

%----------------------------------------------------------------------------------------

% document info command
\newcommand{\documentinfo}[5]{
    \begin{centering}
        \parbox{6.8in}{
        \begin{spacing}{1}
            \begin{flushleft}
                \begin{tabular}{l l} #1 \\ #2 \\ #3 \\ #4 \\ #5 \\
                \end{tabular} \\
                \rule{\textwidth}{1pt}
            \end{flushleft}
        \end{spacing} }
    \end{centering} }

\begin{document}

    \documentinfo{\textbf{MEMO:} TSR-03}{\textbf{DATE: }{\today}}{\textbf{TO: }
    EFC LaBerge}{\textbf{FROM: }\Sabbir, \Jeffrey, \Howard,
    \Brian}{\textbf{SUBJECT: } Team Status Report}
    \vspace{-0.3in}

    \section{Introduction} The Galois Field Arithmetic Unit will accept inputs
    to determine n, and to establish the field generating polynomial. The unit
    would serve as a computation engine for a relatively low-powered
    microcontroller, and would enable complex code and encryption algorithms.
    Project will include implementation of a Reed Solomon encoder and decoder
    using the GFAU. The purpose of this report is to detail the progress of the
    GFAU in the period of November 17, 2017 through December 6, 2017. This is
    the fifth and final status report for the first semester for the GFAU
    team.

    \section{Completed Tasks} During this work period, the team has continued
    to make progress on the GFAU. Including the following achievements:
    \begin{enumerate}[label=\alph*)]

        \item A final high level data flow diagram has been completed.

        \item Our preliminary design review was presented on Wednesday 6th. 

        \item Multiplication module is completed.

        \item Division module is nearing completion.

        \item Term generation low level design is completed.

        \item Incremental progress on the remaining modules has been made.

    \end{enumerate}

    \section{Planned Tasks}
    Before returning from break, our team will:
    \begin{enumerate}[label=\alph*)]


        \item Complete all VHDL code.

        \item Write test benches for all major modules to verify their 
        functionality.

        \item Make a more concrete plan for the the hardware section of our 
        implementation to make sure the second semester goes smoothly.

    \end{enumerate}

    \section{Current Issues} No issues in team dynamics or lack of resources,
    exist in the team so far, however we are behind schedule. This should not
    be an issue however, as we will catch up over winter break, and our work
    load in other classes will be lighter next semester.
   
\end{document}
