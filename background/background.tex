%%%%%%%%%%%%%%%%%%%%%%%%%%%%%%%%%%%%%%%%%
% Template
% LaTeX Template
% Version 1.0 (December 8 2014)
%
% This template has been downloaded from:
% http://www.LaTeXTemplates.com
%
% Original author:
% Brandon Fryslie
% With extensive modifications by:
% Vel (vel@latextemplates.com)
%
% License:
% CC BY-NC-SA 3.0 (http://creativecommons.org/licenses/by-nc-sa/3.0/)
%
% Authors:
% Sabbir Ahmed, Jeffrey Osazuwa, Howard To, Brian Weber
% 
%%%%%%%%%%%%%%%%%%%%%%%%%%%%%%%%%%%%%%%%%

\documentclass[paper=usletter, fontsize=12pt]{article}
%%%%%%%%%%%%%%%%%%%%%%%%%%%%%%%%%%%%%%%%%
% Contract
% Structural Definitions File
% Version 1.0 (December 8 2014)
%
% Created by:
% Vel (vel@latextemplates.com)
% 
% This file has been downloaded from:
% http://www.LaTeXTemplates.com
%
% License:
% CC BY-NC-SA 3.0 (http://creativecommons.org/licenses/by-nc-sa/3.0/)
%
%%%%%%%%%%%%%%%%%%%%%%%%%%%%%%%%%%%%%%%%%

%----------------------------------------------------------------------------------------
%   PARAGRAPH SPACING SPECIFICATIONS
%----------------------------------------------------------------------------------------

\setlength{\parindent}{0mm} % Don't indent paragraphs

\setlength{\parskip}{2.5mm} % Whitespace between paragraphs

%----------------------------------------------------------------------------------------
%   PAGE LAYOUT SPECIFICATIONS
%----------------------------------------------------------------------------------------

\usepackage{geometry} % Required to modify the page layout
\usepackage{multicol}

\setlength{\textwidth}{16cm} % Width of the text on the page
\setlength{\textheight}{23cm} % Height of the text on the page

\setlength{\oddsidemargin}{0cm} % Width of the margin - negative to move text left, positive to move it right

% Uncomment for offset margins if the 'twoside' document class option is used
%\setlength{\evensidemargin}{-0.75cm} 
%\setlength{\oddsidemargin}{0.75cm}

\setlength{\topmargin}{-1.25cm} % Reduce the top margin

%-------------------------------------------

\usepackage[utf8]{inputenc} % Required for including letters with accents
\usepackage[T1]{fontenc} % Use 8-bit encoding that has 256 glyphs

\usepackage{avant} % Use the Avantgarde font for headings
\usepackage{mathptmx} % Use the Adobe Times Roman as the default text font together with math symbols from the Sym­bol, Chancery and Com­puter Modern fonts

%----------------------------------------------------------------------------------------
%   SECTION TITLE SPECIFICATIONS
%----------------------------------------------------------------------------------------

\usepackage{titlesec} % Required for modifying section titles

\titleformat{\section} % Customize the \section{} section title
{\sffamily\large\bfseries} % Title font customizations
{\thesection} % Section number
{16pt} % Whitespace between the number and title
{\large} % Title font size
\titlespacing*{\section}{0mm}{7mm}{0mm} % Left, top and bottom spacing around the title

\titleformat{\subsection} % Customize the \subsection{} section title
{\sffamily\normalsize\bfseries} % Title font customizations
{\thesubsection} % Subsection number
{16pt} % Whitespace between the number and title
{\normalsize} % Title font size
\titlespacing*{\subsection}{0mm}{5mm}{0mm} % Left, top and bottom spacing around the title
\renewcommand\familydefault{\sfdefault} % specifies the document layout and style

% names
\newcommand{\team}{Galois Field Arithmetic Unit}
\newcommand{\Sabbir}{Sabbir Ahmed}
\newcommand{\Jeffrey}{Jeffrey Osazuwa}
\newcommand{\Howard}{Howard To}
\newcommand{\Brian}{Brian Weber}
\newcommand{\polynomial}{$x^{3}+x^{2}+x^{0}$}

% document info command
\newcommand{\documentinfo}[5]{
    \begin{centering}
        \parbox{2in}{
        \begin{spacing}{1}
            \begin{flushleft}
                \begin{tabular}{l l}
                    #1 \\
                    #2 \\
                    #3 \\
                    #4 \\
                    #5 \\
                \end{tabular}\\
                \rule{\textwidth}{1pt}
            \end{flushleft}
        \end{spacing}
        }
    \end{centering}
}

\begin{document}

     \documentinfo{\textbf{MEMO NUMBER:} 03}{\textbf{DATE:} \today}{\textbf{TO: } EFC LaBerge}{\textbf{FROM: }\Sabbir, \Jeffrey, \Howard, \Brian}{\textbf{SUBJECT: } Background on the Galois Field}
    \vspace{-0.1in}

    \section{Background}
    A Galois Field is a field with a finite number of elements. The nomenclature $GF(q)$ is used to indicate a Galois field with q elements. For $GF(q)$ in general, $q$ must be a power of a prime. For each prime power, there exists exactly one finite field. The best known and most used Galois field is $GF(2)$, the binary field.

    \section{Algorithm}
    Input polynomials will be represented as 16 bit arrays, where the coefficient of the terms are represented as 1. The arrays are zero-based, so the 16th bit shall be placed on the 15th index. For example, the polynomial \polynomial~ will be represented as

        \[ <0000 \ 0000 \ 0000 \ 1101> (3rd, \ 2nd, \ and \ 0th \ bits) \]

        \subsection{Determining Irreducibility}
        A polynomial is irreducible if and only if:
        \begin{enumerate}
            \item the coefficient of its 0th term is 1
            \item the total number of non-zero coefficients is odd
        \end{enumerate}

        \subsection{Symbols}

            Once a polynomial is determined irreducible, its symbols may be generated. The number of terms grow exponentially, $2^{n}-1$, where $n$ is the highest degree of the polynomial.

            \subsubsection{Default Symbols}

            Default symbols may be generated concurrently, and consist of all the terms up to $x^{n-1}$.

            \begin{enumerate}
                \item The symbols for $\{x^{0}, x^{1}, \ldots, x^{n-1}\}$ are generated by setting the corresponding bits to 1.
                \item The symbol for $x^{n}$ is generated by setting the corresponding bits for the terms in the polynomial after the highest degree term.
                \item The symbol for $x^{2^{n}-1}$ cycles back to $x^{0}$, and is set to $x^{0}$.
            \end{enumerate}

            Default symbols consist of $n+1$ terms. Therefore, the maximum of 16 bits would have 17 terms generated by default.

           \begin{table}[h]
                \def\arraystretch{2.5}
                \caption{Default Symbols Generated for An Irreducible Polynomial of Degree $\bm{n}$}
                \centering
                \begin{tabular*}{300pt}{@{\extracolsep{\fill}} ccc}

                \textbf{Element} & \textbf{Polynomial Form} & \textbf{Symbol} \\
                \hline
                $0$                 & $0_{n-1} + \ldots + 0_{2} + 0_{1} + 0_{0}$                                           & $\{0_{n-1}\ldots0_{2}0_{1}0_{0}\}$  \\
                $\alpha^{0}$        & $0_{n-1} + \ldots + 0_{2} + 0_{1} + \alpha^{0}_{0}$                                  & $\{0_{n-1}\ldots0_{2}0_{1}1_{0}\}$ \\
                $\alpha^{1}$        & $0_{n-1} + \ldots + 0_{2} + \alpha^{1}_{1} + 0$                                      & $\{0_{n-1}\ldots0_{2}1_{1}0_{0}\}$ \\
                $\alpha^{2}$        & $0_{n-1} + \ldots + \alpha^{2}_{2} + 0_{1} + 0_{0}$                                  & $\{0_{n-1}\ldots1_{2}0_{1}0_{0}\}$ \\
                $\ldots$            & $\ldots$                                                                             & $\ldots$ \\
                $\alpha^{n-1}$      & $1_{n-1} + \ldots + 0_{2} + 0_{1} + 0_{0}$                                           & $\{1_{n-1}\ldots 0_{2}0_{1}0_{0}\}$ \\
                $\alpha^{n}$        & $\alpha^{n-1}_{n-1} + \ldots + \alpha^{2}_{2} + \alpha^{1}_{1} + \alpha^{0}_{0}$     & $\{x_{n-1}\ldots x_{2}x_{1}x_{0}\}$ \\
                $\alpha^{2^{n}-1}$  & $0_{n-1} + \ldots + 0_{2} + 0_{1} + \alpha^{0}_{0}$                                  & $\{0_{n-1}\ldots0_{2}0_{1}1_{0}\}$ \\
                \end{tabular*}
            \end{table}
            \newpage

            \subsubsection{Generated Symbols}
            The rest of the symbols for the terms $x^{n+1}$ to $x^{2^{n}-2}$ must be generated. In total, that would require $2^{n}-2-n-1+1=2^{n}-2-n$ terms. Therefore, the maximum of 16 bits would require 65,518 terms to be generated. \\
            Generating the rest of the symbols may be implemented with a linear feedback shift register (LFSR), using the following recursive equation: \\

                \begin{equation*}
                    \begin{split}
                        \alpha^{n+m} & =\alpha^{n+(m-1)}\times \alpha^{n} \\
                        & = (\alpha^{n+(m-1)} \ll 1 )[n-1] = 1 \Longrightarrow ( \alpha^{n+(m-1)} \ll 1 )[n-2:0] \oplus \alpha^{n}[n-2:0]) \\
                        & \wedge \neg (\alpha^{n+(m-1)} \ll 1 )[n-1] = 1 \Longrightarrow ( \alpha^{n+(m-1)} \ll 1 )[n-2:0] \\
                    \end{split}
                \end{equation*}

            \newpage

        \subsection{Operations}

            \subsubsection{Addition and Subtraction}

            Binary addition and binary subtraction may be done by bitwise XOR-ing the operands.

                \begin{equation*}
                    \begin{split}
                        \alpha^{i} + \alpha^{j} & = \{x^{i}_{n-1}\ldots x^{i}_{2}x^{i}_{1}x^{i}_{0}\} + \{x^{j}_{n-1}\ldots x^{j}_{2}x^{j}_{1}x^{j}_{0}\} \\
                        & = \{(x^{i}_{n-1} \oplus x^{j}_{n-1})\ldots (x^{i}_{2}\oplus x^{j}_{2})(x^{i}_{1}\oplus x^{j}_{1})(x^{i}_{0}\oplus x^{j}_{0})\} \\
                    \end{split}
                \end{equation*}

            \subsubsection{Multiplication and Division}

            \subsubsection{Logarithm}

            \newpage

    \section{VHSIC Hardware Design Language (VHDL) Implementation}
    TODO:

    \section{Example}
        \subsection{Show that \polynomial~ is irreducible in $GF(2)[x]$}

            \[ x=0: (0)^{3}+(0)^{2}+(0)^{0}=0+0+1=1 \ (not \ a \ root) \]
            \[ x=1: (1)^{3}+(1)^{2}+(1)^{0}=1+1+1=1 \ (not \ a \ root) \]
            \centerline{$\therefore$ \polynomial~ has no roots in $GF(2)[x]$.}

        \subsection{Generate the 8 elements of $GF(2^{3})$ using the primitive polynomial \polynomial~.}

            \centerline{Let $\beta \ \epsilon \ GF(2^{3})$ be a root of \polynomial~ $\implies \beta^{3}+\beta^{2}+\beta^{0}$}

            \hspace*{\fill}
            \centerline{$\therefore$ The coefficients are in $GF(2) \implies \beta^{3}=\beta^{2}+\beta^{0}$}

            \[ \because a \ field \ has \ additive \ and \ multiplicative \ identities: \]
            \[ \therefore 0, 1=\beta^{0} \ \epsilon \ GF(2^{3}) \]

            \[ \therefore \beta^{1} \ \epsilon \ GF(2^{3}) \ (\because \ closure \ of \ multiplication) \]

            \[ \therefore \beta^{2} \ \epsilon \ GF(2^{3}) \ (\because \ assumption) \]

            \[ \therefore \beta^{3} \ \epsilon \ GF(2^{3}) \ (\because \beta^{3}=\beta^{2}+\beta^{0}) \]

            \begin{equation*}
                \begin{split}
                    \because \beta^{4} & = \beta^{1} \times \beta^{3} \\
                    & = \beta^{1} (\beta^{2}+\beta^{0}) \\
                    & = \beta^{3}+\beta^{1} \\
                    & = \beta^{2}+\beta^{1}+\beta^{0}
                \end{split}
            \end{equation*}

            \[ \therefore \beta^{4} \ \epsilon \ GF(2^{3}) \]

            \begin{equation*}
                \begin{split}
                    \because \beta^{5} & = \beta^{1} \times \beta^{4} \\
                    & = \beta^{1} (\beta^{2}+\beta^{1}+\beta^{0}) \\
                    & = \beta^{3}+\beta^{2}+\beta^{1} \\
                    & = \beta^{2}+\beta^{0}+\beta^{2}+\beta^{1} \\
                    & = \beta^{1}+\beta^{0}
                \end{split}
            \end{equation*}

            \[ \therefore \beta^{5} \ \epsilon \ GF(2^{3}) \]

            \begin{equation*}
                \begin{split}
                    \because \beta^{6} & = \beta^{1} \times \beta^{5} \\
                    & = \beta^{1} (\beta^{1}+\beta^{0}) \\
                    & = \beta^{2}+\beta^{1}
                \end{split}
            \end{equation*}
            \[ \therefore \beta^{6} \ \epsilon \ GF(2^{3}) \]

            \begin{equation*}
                \begin{split}
                    \because \beta^{7} & = \beta^{1} \times \beta^{6} \\
                    & = \beta^{1} (\beta^{2}+\beta^{1}) \\
                    & = \beta^{3}+\beta^{2} \\
                    & = \beta^{2}+\beta^{0}+\beta^{2} \\
                    & = \beta^{0} = 1
                \end{split}
            \end{equation*}
            \[ \therefore \beta^{7} \ \epsilon \ GF(2^{3}) \]

            \begin{table}[h]
                \caption{The 8 Element Vectors of \polynomial~ in $GF(2)[x]$}

                \centering
                \begin{tabular*}{200pt}{@{\extracolsep{\fill}} c | c | c}

                \textbf{Element} & \textbf{Polynomial Form} & \textbf{Symbol} \\
                \hline
                $0$           & $0+0+0$                                 & 000 \\
                $\beta^{0}$ & $0 + 0 + \beta^{0}$                   & 001 \\
                $\beta^{1}$ & $0 + \beta^{1} + 0$                   & 010 \\
                $\beta^{2}$ & $\beta^{2} + 0 + 0$                   & 100 \\
                $\beta^{3}$ & $\beta^{2} + 0 + \beta^{0}$           & 101 \\
                $\beta^{4}$ & $\beta^{2} + \beta^{1} + \beta^{0}$   & 111 \\
                $\beta^{5}$ & $0 + \beta^{1} + \beta^{0}$           & 011 \\
                $\beta^{6}$ & $\beta^{2} + \beta^{1} + 0$           & 110 \\
                $\beta^{7}$ & $0 + 0 + \beta^{0}$                   & 001 \\
                \end{tabular*}
            \end{table}

        \newpage
        \subsection{Generate the addition (bitwise XOR) and multiplication tables for the implementation of $GF(2)[x]$}

            \begin{table}[h]
                \def\arraystretch{1.5}
                \caption{Addition Table for \polynomial~ in $GF(2)[x]$}
                \centering
                \begin{tabular*}{250pt}{@{\extracolsep{\fill}} c | c c c c c c c c}

                    $\bm{+}$ & $\bm{0}$ & $\bm{\beta^{0}}$ & $\bm{\beta^{1}}$ & $\bm{\beta^{2}}$ 
                    & $\bm{\beta^{3}}$ & $\bm{\beta^{4}}$ & $\bm{\beta^{5}}$ & $\bm{\beta^{6}}$ \\
                    \hline

                    $\bm{0}$ & $0$ & $\beta^{0}$ & $\beta^{1}$ & $\beta^{2}$ 
                    & $\beta^{3}$ & $\beta^{4}$ & $\beta^{5}$ & $\beta^{6}$ \\

                    $\bm{\beta^{0}}$ & $\beta^{0}$ & $0$ & $\beta^{5}$ & $\beta^{3}$ 
                    & $\beta^{2}$ & $\beta^{6}$ & $\beta^{1}$ & $\beta^{4}$ \\

                    $\bm{\beta^{1}}$ & $\beta^{1}$ & $\beta^{5}$ & $0$ & $\beta^{6}$ 
                    & $\beta^{4}$ & $\beta^{3}$ & $\beta^{0}$ & $\beta^{2}$ \\

                    $\bm{\beta^{2}}$ & $\beta^{2}$ & $\beta^{3}$ & $\beta^{6}$ & $0$ 
                    & $\beta^{0}$ & $\beta^{5}$ & $\beta^{4}$ & $\beta^{1}$ \\

                    $\bm{\beta^{3}}$ & $\beta^{3}$ & $\beta^{2}$ & $\beta^{4}$ & $\beta^{0}$ 
                    & $0$ & $\beta^{1}$ & $\beta^{6}$ & $\beta^{5}$ \\

                    $\bm{\beta^{4}}$ & $\beta^{4}$ & $\beta^{6}$ & $\beta^{3}$ & $\beta^{5}$ 
                    & $\beta^{1}$ & $0$ & $\beta^{2}$ & $\beta^{0}$ \\

                    $\bm{\beta^{5}}$ & $\beta^{5}$ & $\beta^{1}$ & $\beta^{0}$ & $\beta^{4}$ 
                    & $\beta^{6}$ & $\beta^{2}$ & $0$ & $\beta^{3}$ \\

                    $\bm{\beta^{6}}$ & $\beta^{6}$ & $\beta^{4}$ & $\beta^{2}$ & $\beta^{1}$ 
                    & $\beta^{5}$ & $\beta^{0}$ & $\beta^{3}$ & $0$ \\

                \end{tabular*}
            \end{table}

            \begin{table}[h]
                \def\arraystretch{1.5}
                \caption{Multiplication Table for \polynomial~ in $GF(2)[x]$}
                \centering
                \begin{tabular*}{250pt}{@{\extracolsep{\fill}} c | c c c c c c c c}

                    \bm{$\times$} & $\bm{0}$ & $\bm{\beta^{0}}$ & $\bm{\beta^{1}}$ & $\bm{\beta^{2}}$ 
                    & $\bm{\beta^{3}}$ & $\bm{\beta^{4}}$ & $\bm{\beta^{5}}$ & $\bm{\beta^{6}}$ \\
                    \hline

                    $\bm{0}$ & $0$ & $0$ & $0$ & $0$ & $0$ & $0$ & $0$ & $0$ \\

                    $\bm{\beta^{0}}$ & $0$ & $\beta^{0}$ & $\beta^{1}$ & $\beta^{2}$ 
                    & $\beta^{3}$ & $\beta^{4}$ & $\beta^{5}$ & $\beta^{6}$ \\

                    $\bm{\beta^{1}}$ & $0$ & $\beta^{1}$ & $\beta^{2}$ & $\beta^{3}$ 
                    & $\beta^{4}$ & $\beta^{5}$ & $\beta^{6}$ & $\beta^{0}$ \\

                    $\bm{\beta^{2}}$ & $0$ & $\beta^{2}$ & $\beta^{3}$ & $\beta^{4}$ 
                    & $\beta^{5}$ & $\beta^{6}$ & $\beta^{0}$ & $\beta^{1}$ \\

                    $\bm{\beta^{3}}$ & $0$ & $\beta^{3}$ & $\beta^{4}$ & $\beta^{5}$ 
                    & $\beta^{6}$ & $\beta^{0}$ & $\beta^{1}$ & $\beta^{2}$ \\

                    $\bm{\beta^{4}}$ & $0$ & $\beta^{4}$ & $\beta^{5}$ & $\beta^{6}$ 
                    & $\beta^{0}$ & $\beta^{1}$ & $\beta^{2}$ & $\beta^{3}$ \\

                    $\bm{\beta^{5}}$ & $0$ & $\beta^{5}$ & $\beta^{6}$ & $\beta^{0}$ 
                    & $\beta^{1}$ & $\beta^{2}$ & $\beta^{3}$ & $\beta^{4}$ \\

                    $\bm{\beta^{6}}$ & $0$ & $\beta^{6}$ & $\beta^{0}$ & $\beta^{1}$ 
                    & $\beta^{2}$ & $\beta^{3}$ & $\beta^{4}$ & $\beta^{5}$ \\

                \end{tabular*}
            \end{table}

\end{document}
