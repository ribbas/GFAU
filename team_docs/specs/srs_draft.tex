%%%%%%%%%%%%%%%%%%%%%%%%%%%%%%%%%%%%%%%%%
% Template
% LaTeX Template
% Version 1.0 (December 8 2014)
%
% This template has been downloaded from:
% http://www.LaTeXTemplates.com
%
% Original author:
% Brandon Fryslie
% With extensive modifications by:
% Vel (vel@latextemplates.com)
%
% License:
% CC BY-NC-SA 3.0 (http://creativecommons.org/licenses/by-nc-sa/3.0/)
%
% Authors:
% Sabbir Ahmed, Jeffrey Osazuwa, Howard To, Brian Weber
% 
%%%%%%%%%%%%%%%%%%%%%%%%%%%%%%%%%%%%%%%%%

\documentclass[paper=usletter, fontsize=12pt]{article}
%%%%%%%%%%%%%%%%%%%%%%%%%%%%%%%%%%%%%%%%%
% Structure Structural Definitions File Version 1.0 (December 8 2014)
%
% Created by: Vel (vel@latextemplates.com)
% 
% This file has been downloaded from: http://www.LaTeXTemplates.com
%
% License: CC BY-NC-SA 3.0 (http://creativecommons.org/licenses/by-nc-sa/3.0/)
%
%%%%%%%%%%%%%%%%%%%%%%%%%%%%%%%%%%%%%%%%%

\usepackage{geometry}  % page layout modification

\usepackage[utf8]{inputenc}  % letters with accents
\usepackage[T1]{fontenc}  % use 8-bit encoding that has 256 glyphs
\usepackage{avant}  % Avantgarde font

\usepackage{amsmath}  % equations
\usepackage{amssymb}  % extention of symbols
\usepackage{bm}  % extention of bolding features
\usepackage{caption}  % captions for tables
\usepackage{enumitem}  % extra options for enumeration
\usepackage[pdftex]{graphicx}  % figures
\usepackage{setspace}  % extension of line spacing options

\usepackage{enumitem}

\usepackage[square,numbers]{natbib}  % for bibtex

% add more sections beyond \subsubsection
\setcounter{tocdepth}{4}
\setcounter{secnumdepth}{4}

\setlength{\parindent}{0mm} % don't indent paragraphs
\setlength{\parskip}{2.5mm} % space between paragraphs
\renewcommand{\baselinestretch}{1.5}  % space between lines

\setlength{\textwidth}{16cm} % width of the text on the page
\setlength{\textheight}{23cm} % height of the text on the page
\setlength{\oddsidemargin}{0cm} % width of the margin
\setlength{\topmargin}{-1.25cm} % reduce the top margin

\captionsetup[table]{skip=10pt}  % space between tables and their captions
\captionsetup{labelfont=bf}  % bold captions

\renewcommand\familydefault{\sfdefault}  % default font for entire document
 % specifies the document layout and style

%-----------------------------------------------------------------------------

% names
\newcommand{\team}{Galois Field Arithmetic Unit}
\newcommand{\Sabbir}{Sabbir Ahmed}
\newcommand{\Jeffrey}{Jeffrey Osazuwa}
\newcommand{\Howard}{Howard To}
\newcommand{\Brian}{Brian Weber}

% document info command
\newcommand{\documentinfo}[5]{
    \begin{centering}
        \parbox{2in}{
        \begin{spacing}{1}
            \begin{flushleft}
                \begin{tabular}{l l}
                    #1 \\
                    #2 \\
                    #3 \\
                    #4 \\
                    #5 \\
                \end{tabular}\\
                \rule{\textwidth}{1pt}
            \end{flushleft}
        \end{spacing}
        }
    \end{centering}
}

\begin{document}


    \documentinfo
    {\textbf{MEMO NUMBER:} GFAU\_SRS\_01}
    {\textbf{DATE:} \today}
    {\textbf{TO: } EFC LaBerge}
    {\textbf{FROM: }\Sabbir, \Jeffrey, \Howard, \Brian}
    {\textbf{SUBJECT: } System Requirement Specifications Draft}
    \vspace{-0.1in}

    \section{Introduction}
    A Galois field is a field with a finite number of elements. The nomenclature $GF(q)$ is used to indicate a Galois field with q elements. For $GF(q)$ in general, $q$ must be a power of a prime. For each prime power, there exists exactly one finite field. The best known and most used Galois field is $GF(2)$, the binary field.

    The \team~ handles irreducible polynomials in $GF(2^n)$, where $\{2 \leq n \leq 16\}$. The ALU generates all the terms in the field of the polynomial, and allows the user to view and apply the following binary operations:

    \begin{itemize}

        \item Addition
        \item Subtraction
        \item Multiplication
        \item Division
        \item Logarithm

    \end{itemize}

        \subsection{Document Overview}
        This document serves as the System Requirements Specification for the Galois Field Arithmetic Unit. The description and requirements of the project are embodied in this document.

        The Specification is divided into separate segments pertaining to individual components and requirements on different levels. Figures and tables are attached where necessary to assist in demonstrating concepts.

        \subsection{Mission Scenario}
        Cryptography has many expensive calculations that are difficult for low power and inexpensive microcontrollers to handle. Galois fields are frequently used in the field of cryptography. The GFAU will make Galois field calculations more accessible to such low powered devices.

        \subsection{System Overview}
        The GFAU system is composed of discrete modules residing in a single programmable board. Individual modules shall be programmed to solely complete an assigned task. Although modules are assigned individual tasks, they shall not have exclusive components for its functionality.

        \subsection{System Boundary Diagram}
        The system boundary diagram of the GFAU has been provided in Figure \ref{fig:system_boundary} below.

        \begin{figure}[ht]
            \begin{center}
                \includegraphics[width=1\textwidth]{system_boundary.png}
                \caption{System Boundary Diagram of the \team~, where $n$ is the Number of Terms} \label{fig:system_boundary}
            \end{center}
        \end{figure}

        The user input and output (I/O) interface are handled by the microcontroller (MCU) which are transferred via busses. The user inputs consist of the mode bit, the input generating polynomial and the binary operation(s) along with their corresponding operands. The MCU transfers the data to the unit to perform the desired operations. The GFAU will return the outputs and any errors detected back to the MCU.
        \newpage

        \subsection{Functional Flow Diagram}
        The functional flow diagram of the GFAU has been provided in Figure \ref{fig:functional_flow}.

        \begin{figure}[ht]
            \begin{center}
                \includegraphics[width=1\textwidth]{functional_flow.png}
                \caption{Functional Flow Diagram of the \team~} \label{fig:functional_flow}
            \end{center}
        \end{figure}

        The diagram provides a high-level overview of the sequence of processes that take place in the unit. In total, the unit waits for an input from the user in three separate instances. The order of the inputs are essential for the unit to proceed as desired. The \textbf{mode} input sets the width of the data bus in the GFAU. The \textbf{polynomial} input consist of the generating polynomial to generate its terms in the Galois field. The \textbf{operation} input consists of the two operands along with the desired binary operation.
        \newpage

        \subsection{Data Flow Diagram}
        The data flow diagram of the GFAU has been provided in Figure \ref{fig:data_flow}.

        \begin{figure}[ht]
            \begin{center}
                \includegraphics[width=1\textwidth]{data_flow.png}
                \caption{Data Flow Diagram of the \team~} \label{fig:data_flow}
            \end{center}
        \end{figure}

        The diagram provides a lower-level view of the system emphasizing the individual components and their role in converting the input data to the desired output. The \textbf{Error} lines are used by multiple components to send interrupt signals to the user when required.

    \section{Requirements}

        \subsection{System Requirements}

        \subsection{Hardware Requirements}
        Hardware portability shall be prioritized on the GFAU. The hardware portability provides flexibility in the ranges of its specifications. The ranges of these specifications shall be reasonably bounded.

        \begin{enumerate}

            \item The unit shall be functional at a variety of clock speeds at a minimum range of [4 MHz-100 MHz].

            \item The GFAU shall perform its computations in [] instructions per clock

            \item ...

        \end{enumerate}

        \subsection{Software and Testing Requirements}
        Extensive simulations shall be conducted during the development of the design in a hardware description language (HDL). This process makes debugging easier and minimizes the risk of unintended behavior in the prototype. This section outlines the HDL code requirements and tests that the simulations are required to pass.

        \begin{enumerate}

            \item Extensive design verification simulations shall be done before purchasing any hardware

            \item All HDL code shall be synthesizable

            \item Gate delay and other relevant values shall be parametrized to easily match the specifications of candidate hardware during verification

        \end{enumerate}

        \subsection{Communication Requirements}
        In order for the GFAU to be able to communicated with a wide variety of external devices, the GFAU shall implement communication methods which are commonly used. The details of how the GFAU shall communicate are outlined in this section.

        \begin{enumerate}
            
            \item The GFAU shall have an option to use either a 8, 16, [or 32] bit data-bus.
            
            \item After any given operation is complete and the result is placed on the bus, the GFAU shall set a ready pin. 
            
            \item The external device may monitor the ready pin using polling or interrupts to know when to pull the data from the bus.
            
            \item The GFAU shall use a common or easy to implement protocol for pushing blocks of data over a bus that is too small to send on one clock.

        \end{enumerate}

            \subsubsection{Errors}
            This section outlines the cases in which errors are thrown.
            \begin{enumerate}

                \item The unit shall produce an error if an operand is not in the same Galois field.

                \item The unit shall produce an error if the FPGA does not receive all the inputs from the microcontroller.

                \item The microcontroller shall produce an error if it does not receive all the outputs from the FPGA.

            \end{enumerate}


\end{document}
