%%%%%%%%%%%%%%%%%%%%%%%%%%%%%%%%%%%%%%%%%
% Template
% LaTeX Template
% Version 1.0 (December 8 2014)
%
% This template has been downloaded from:
% http://www.LaTeXTemplates.com
%
% Original author:
% Brandon Fryslie
% With extensive modifications by:
% Vel (vel@latextemplates.com)
%
% License:
% CC BY-NC-SA 3.0 (http://creativecommons.org/licenses/by-nc-sa/3.0/)
%
% Authors:
% Sabbir Ahmed, Jeffrey Osazuwa, Howard To, Brian Weber
% 
%%%%%%%%%%%%%%%%%%%%%%%%%%%%%%%%%%%%%%%%%

\documentclass[paper=usletter, fontsize=12pt]{article}
%%%%%%%%%%%%%%%%%%%%%%%%%%%%%%%%%%%%%%%%%
% Structure Structural Definitions File Version 1.0 (December 8 2014)
%
% Created by: Vel (vel@latextemplates.com)
% 
% This file has been downloaded from: http://www.LaTeXTemplates.com
%
% License: CC BY-NC-SA 3.0 (http://creativecommons.org/licenses/by-nc-sa/3.0/)
%
%%%%%%%%%%%%%%%%%%%%%%%%%%%%%%%%%%%%%%%%%

\usepackage{geometry}  % page layout modification

\usepackage[utf8]{inputenc}  % letters with accents
\usepackage[T1]{fontenc}  % use 8-bit encoding that has 256 glyphs
\usepackage{avant}  % Avantgarde font

\usepackage{amsmath}  % equations
\usepackage{amssymb}  % extention of symbols
\usepackage{bm}  % extention of bolding features
\usepackage{caption}  % captions for tables
\usepackage{enumitem}  % extra options for enumeration
\usepackage[pdftex]{graphicx}  % figures
\usepackage{setspace}  % extension of line spacing options

\usepackage{enumitem}

\usepackage[square,numbers]{natbib}  % for bibtex

% add more sections beyond \subsubsection
\setcounter{tocdepth}{4}
\setcounter{secnumdepth}{4}

\setlength{\parindent}{0mm} % don't indent paragraphs
\setlength{\parskip}{2.5mm} % space between paragraphs
\renewcommand{\baselinestretch}{1.5}  % space between lines

\setlength{\textwidth}{16cm} % width of the text on the page
\setlength{\textheight}{23cm} % height of the text on the page
\setlength{\oddsidemargin}{0cm} % width of the margin
\setlength{\topmargin}{-1.25cm} % reduce the top margin

\captionsetup[table]{skip=10pt}  % space between tables and their captions
\captionsetup{labelfont=bf}  % bold captions

\renewcommand\familydefault{\sfdefault}  % default font for entire document
 % specifies the document layout and style

%----------------------------------------------------------------------------------------

% names
\newcommand{\team}{Galois Field Arithmetic Unit}
\newcommand{\Sabbir}{Sabbir Ahmed}
\newcommand{\Jeffrey}{Jeffrey Osazuwa}
\newcommand{\Howard}{Howard To}
\newcommand{\Brian}{Brian Weber}

% document info command
\newcommand{\documentinfo}[5]{
    \begin{centering}
        \parbox{2in}{
        \begin{spacing}{1}
            \begin{flushleft}
                \begin{tabular}{l l}
                    #1 \\
                    #2 \\
                    #3 \\
                    #4 \\
                    #5 \\
                \end{tabular}\\
                \rule{\textwidth}{1pt}
            \end{flushleft}
        \end{spacing}
        }
    \end{centering}
}

\begin{document}


    \documentinfo{\textbf{MEMO NUMBER:} 03}{\textbf{DATE:} \today}{\textbf{TO: } EFC LaBerge}{\textbf{FROM: }\Sabbir, \Jeffrey, \Howard, \Brian}{\textbf{SUBJECT: } System Requirement Specifications Draft}
    \vspace{-0.1in}

    \section{Introduction}
    A Galois field is a field with a finite number of elements. The nomenclature $GF(q)$ is used to indicate a Galois field with q elements. For $GF(q)$ in general, $q$ must be a power of a prime. For each prime power, there exists exactly one finite field. The best known and most used Galois field is $GF(2)$, the binary field.

    The \team~ handles irreducible polynomials in $GF(2^n)$, where $\{2 \leq n \leq 16\}$. The ALU generates all the terms in the field of the polynomial, and allows the user to view and apply addition, subtraction, multiplication, division or logarithm between them.

        \subsection{Document Overview}

        \subsection{System Overview}

        \subsection{Mission Scenario}
        Cryptography has many expensive calculations that are difficult for low power and inexpensive microcontroller units to handle. The GFAU will make Galois field calculations more accessible to these low power devices.

        \subsection{System Boundary Diagram}
        \begin{figure}[ht]
            \begin{center}
                \includegraphics[width=1\textwidth]{system_boundary.png}
                \caption{System Boundary Diagram of the \team~, where $n$ is the Number of Terms} \label{fig:system_boundary}
            \end{center}
        \end{figure}

        \subsection{Data Flow Diagram}
        \begin{figure}[ht]
            \begin{center}
                \includegraphics[width=1\textwidth]{data_flow.png}
                \caption{Data Flow Diagram of the \team~} \label{fig:data_flow}
            \end{center}
        \end{figure}


        \subsection{Functional Flow Diagram}


    \section{Requirements}

        \subsection{System Requirements}

        \subsection{Hardware Requirements}


\end{document}
