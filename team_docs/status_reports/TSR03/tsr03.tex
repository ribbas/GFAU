%%%%%%%%%%%%%%%%%%%%%%%%%%%%%%%%%%%%%%%%%
% Contract LaTeX Template Version 1.0 (December 8 2014)
%
% This template has been downloaded from: http://www.LaTeXTemplates.com
%
% Original author: Brandon Fryslie With extensive modifications by: Vel
% (vel@latextemplates.com)
%
% License: CC BY-NC-SA 3.0 (http://creativecommons.org/licenses/by-nc-sa/3.0/)
%
% Authors: Sabbir Ahmed, Jeffrey Osazuwa, Howard To, Brian Weber
% 
%%%%%%%%%%%%%%%%%%%%%%%%%%%%%%%%%%%%%%%%%

\documentclass[paper=usletter, fontsize=12pt]{article}
%%%%%%%%%%%%%%%%%%%%%%%%%%%%%%%%%%%%%%%%%
% Structure
% Structural Definitions File
% Version 1.0 (December 8 2014)
%
% Created by:
% Vel (vel@latextemplates.com)
% 
% This file has been downloaded from:
% http://www.LaTeXTemplates.com
%
% License:
% CC BY-NC-SA 3.0 (http://creativecommons.org/licenses/by-nc-sa/3.0/)
%
%%%%%%%%%%%%%%%%%%%%%%%%%%%%%%%%%%%%%%%%%

\usepackage{geometry} % Required to modify the page layout

\usepackage{amsmath}
\usepackage{amssymb}

\usepackage[utf8]{inputenc} % Required for including letters with accents
\usepackage[T1]{fontenc} % Use 8-bit encoding that has 256 glyphs

\usepackage{avant} % Use the Avantgarde font for headings
\usepackage{setspace}

\setlength{\parindent}{0mm} % Don't indent paragraphs
\setlength{\parskip}{2.5mm} % Whitespace between paragraphs

\setlength{\textwidth}{16cm} % Width of the text on the page
\setlength{\textheight}{23cm} % Height of the text on the page
\setlength{\oddsidemargin}{0cm} % Width of the margin - negative to move text left, positive to move it right
\setlength{\topmargin}{-1.25cm} % Reduce the top margin

\renewcommand\familydefault{\sfdefault}  % default font for entire document
 % specifies the document layout and style

\newcommand{\team}{Galois Field Arithmetic Unit}
\newcommand{\Sabbir}{Sabbir Ahmed}
\newcommand{\Jeffrey}{Jeffrey Osazuwa}
\newcommand{\Howard}{Howard To}
\newcommand{\Brian}{Brian Weber}

%----------------------------------------------------------------------------------------

% document info command
\newcommand{\documentinfo}[5]{
    \begin{centering}
        \parbox{6.8in}{
        \begin{spacing}{1}
            \begin{flushleft}
                \begin{tabular}{l l} #1 \\ #2 \\ #3 \\ #4 \\ #5 \\
                \end{tabular} \\
                \rule{\textwidth}{1pt}
            \end{flushleft}
        \end{spacing} }
    \end{centering} }

\begin{document}

    \documentinfo{\textbf{MEMO:} TSR-03}{\textbf{DATE: }{\today}}{\textbf{TO: }
    EFC LaBerge}{\textbf{FROM: }\Sabbir, \Jeffrey, \Howard,
    \Brian}{\textbf{SUBJECT: } Team Status Report}
    \vspace{-0.3in}

    \section{Introduction} The Galois Field Arithmetic Unit will accept inputs
    to determine n, and to establish the field generating polynomial. The unit
    would serve as a computation engine for a relatively low-powered
    microcontroller, and would enable complex code and encryption algorithms.
    Project will include implementation of a Reed Solomon encoder and decoder
    using the GFAU. The purpose of this report is to detail the progress of the
    GFAU in the period of October 20, 2017 through November 3, 2017. This is
    the third status report for the GFAU project.

    \section{Completed Tasks} During this work period, the team has continued
    to make progress on the GFAU. Including the following achievements:
    \begin{enumerate}[label=\alph*)]

        \item Several low level schematics have been designed for the system.

        \item Multiple low level schematics have been integrated together to
        validate their data and functional flow.

        \item Development of the VHDL modules for the operation subsystems has
        begun. Each members of the team has been assigned separate modules to
        take responsibility of.

        \item Several algorithms have been considered for the multiplication
        and division operations and their trade-offs were studied.

        \item The team has decided on utilizing two external memory chips to
        store the lookup tables of the generated terms. Parallel memory chips
        should allow more convenient conversions between terms and their
        symbolic and polynomial values.

        \item A Statement of Work detailing the schedule and milestones of the
        project was developed and submitted to Dr. LaBerge.

    \end{enumerate}

    \section{Planned Tasks}
    \begin{enumerate}[label=\alph*)]

        \item Decide on the best algorithms for multiplication and division of
        Galois operands.

        \item Polish and finalize the low level schematics which have VHDL
        modules associated with them.

        \item Conduct further research on accessing data in external parallel
        memory chips.

        \item Meet with Dr. Robucci or Dr. Mohsenin to discuss features of
        development boards and their trade-offs.

    \end{enumerate}

    \section{Current Issues} No issues in team dynamics or lack of resources,
    exist in the team so far.
   
\end{document}
