%%%%%%%%%%%%%%%%%%%%%%%%%%%%%%%%%%%%%%%%%
% Template LaTeX Template Version 1.0 (December 8 2014)
%
% This template has been downloaded from: http://www.LaTeXTemplates.com
%
% Original author: Brandon Fryslie With extensive modifications by: Vel
% (vel@latextemplates.com)
%
% License: CC BY-NC-SA 3.0 (http://creativecommons.org/licenses/by-nc-sa/3.0/)
%
% Authors: Sabbir Ahmed, Jeffrey Osazuwa, Howard To, Brian Weber
% 
%%%%%%%%%%%%%%%%%%%%%%%%%%%%%%%%%%%%%%%%%

\documentclass[paper=usletter, fontsize=12pt]{article}
%%%%%%%%%%%%%%%%%%%%%%%%%%%%%%%%%%%%%%%%%
% Structure Structural Definitions File Version 1.0 (December 8 2014)
%
% Created by: Vel (vel@latextemplates.com)
% 
% This file has been downloaded from: http://www.LaTeXTemplates.com
%
% License: CC BY-NC-SA 3.0 (http://creativecommons.org/licenses/by-nc-sa/3.0/)
%
%%%%%%%%%%%%%%%%%%%%%%%%%%%%%%%%%%%%%%%%%

\usepackage{geometry}  % page layout modification

\usepackage[utf8]{inputenc}  % letters with accents
\usepackage[T1]{fontenc}  % use 8-bit encoding that has 256 glyphs
\usepackage{avant}  % Avantgarde font

\usepackage{amsmath}  % equations
\usepackage{amssymb}  % extention of symbols
\usepackage{bm}  % extention of bolding features
\usepackage{caption}  % captions for tables
\usepackage{enumitem}  % extra options for enumeration
\usepackage[pdftex]{graphicx}  % figures
\usepackage{setspace}  % extension of line spacing options

\usepackage{enumitem}

\usepackage[square,numbers]{natbib}  % for bibtex

% add more sections beyond \subsubsection
\setcounter{tocdepth}{4}
\setcounter{secnumdepth}{4}

\setlength{\parindent}{0mm} % don't indent paragraphs
\setlength{\parskip}{2.5mm} % space between paragraphs
\renewcommand{\baselinestretch}{1.5}  % space between lines

\setlength{\textwidth}{16cm} % width of the text on the page
\setlength{\textheight}{23cm} % height of the text on the page
\setlength{\oddsidemargin}{0cm} % width of the margin
\setlength{\topmargin}{-1.25cm} % reduce the top margin

\captionsetup[table]{skip=10pt}  % space between tables and their captions
\captionsetup{labelfont=bf}  % bold captions

\renewcommand\familydefault{\sfdefault}  % default font for entire document
 % specifies the document layout and style

%----------------------------------------------------------------------------------------

% names
\newcommand{\team}{Galois Field Arithmetic Unit}
\newcommand{\Sabbir}{Sabbir Ahmed}
\newcommand{\Jeffrey}{Jeffrey Osazuwa}
\newcommand{\Howard}{Howard To}
\newcommand{\Brian}{Brian Weber}

% document info command
\newcommand{\documentinfo}[5]{
    \begin{centering}
        \parbox{6.8in}{
        \begin{spacing}{1}
            \begin{flushleft}
                \begin{tabular}{l l} #1 \\ #2 \\ #3 \\ #4 \\ #5 \\
                \end{tabular}\\
                \rule{\textwidth}{1pt}
            \end{flushleft}
        \end{spacing} }
    \end{centering} }

\begin{document}

    \documentinfo {\textbf{MEMO NUMBER:} GFAU-SOW} {\textbf{DATE:} {\today}}
    {\textbf{TO: } EFC LaBerge} {\textbf{FROM: }\Sabbir, \Jeffrey, \Howard,
    \Brian} {\textbf{SUBJECT: } Galois Field Arithmetic Unit Statement of Work}
    \vspace{-0.3in}

    \section{Introduction} A Galois field is a field with a finite number of
    elements. The nomenclature $GF(q)$ is used to indicate a Galois field with
    q elements. For $GF(q)$ in general, $q$ must be a power of a prime. For
    each prime power, there exists exactly one finite field. The best known and
    most used Galois field is $GF(2)$, the binary field.

    The \team~ handles irreducible polynomials in $GF(2^n)$, where $\{2 \leq n
    \leq 16\}$. The ALU generates all the terms in the field of the polynomial,
    and allows the user to view and apply the following binary operations:

    \begin{itemize}

        \item Addition
        \item Subtraction
        \item Multiplication
        \item Division
        \item Logarithm

    \end{itemize}

        \subsection{Purpose and Scope} This Statement of Work outlines and
        elaborates the tasks necessary to implement the project. The document
        also details their corresponding milestones and deadlines and how the
        contribution will be divided within the team.

    \section{Roles and Division of Labor} This project requires equal team work
on all tasks because of the steep learning curve on implementing coprocessors
with programmable boards. None of the team members have comprehensive prior
knowledge or training on field programmable gate array units, and are therefore
required to learn the concepts concurrently.

Although there is no clear division of labor within the team, each members have been implicitly designated unofficial roles.

Howard has been chosen to act as the point of contact between the team and the project manager and other consultants. He has also been trusted with the scheduling of team meetings and milestones.

Sabbir is responsible for validating the system inputs and outputs through the operations in the unit. He is also accountable for providing background information on the Galois field and its operations related to the arithmetic unit.

Brian has been put in charge with the digital design of the system at various levels. He is also responsible for overseeing the designing of the system using the hardware description language.

Jeffrey supports the designing processes by providing test benches and by synthesizing the individual modules.

Each members of the GFAU shall contribute to designing the individual modules and the entire system.

    \section{Tasks}

        \subsection{Research}

        \subsubsection{Background}

        \subsubsection{Devices}

            \paragraph{Field Programmable Gate Array (FPGA)}

            \paragraph{External Devices}

        \subsection{Design}

            \subsubsection{System Boundary}

            \subsubsection{Schematics}

        \subsection{Software Implementation}

            \subsubsection{Design in VHSIC Hardware Description Language (VHDL)}

            \subsubsection{Simulation and Synthesis in VHDL}

            \subsubsection{External Devices}

        \subsection{Purchases}

        \subsection{Hardware}

            \subsubsection{Integration}

    \section{Deliverables}

\end{document}
